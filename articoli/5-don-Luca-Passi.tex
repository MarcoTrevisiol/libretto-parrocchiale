\section{Nel nome di don luca passi, un dono reciproco, guardando oltre}


Con queste parole possiamo riassumere quanto si è verificato il 13 aprile 2024 a Maserada sul Piave, con l’esperienza del Recital in memoria del Beato Luca Passi.

Ma lasciamo la parola al gruppo recital di Schignano che mi sembra sintetizzi benissimo questa esperienza.

\figurawrap{0.5\textwidth}{auditorium.jpg}

``Abbiamo ricaricato le pile dell’anima!'': questo in sintesi il racconto dell’esperienza dei volontari del “Gruppo recital Schignano”, in trasferta con una delegazione fino a Maserada del Piave (TV) lo scorso 13 aprile, dalle suore maestre di Santa Dorotea in occasione del decennale (+1) di beatificazione di don Luca Passi. ``Quando ci incontrammo per la prima volta a Venezia con la Madre Superiora, scattò quel feeling, il capirsi al primo sguardo: in questi 26 anni di attività abbiamo sempre fondato il nostro ''piccolo fare del bene`` e recitare sul valore dell’amicizia e sulla carità. Partiti come ''Gruppo giovani dell’oratorio`` nel lontano 1998, ci siamo riconosciuti subito nel messaggio di don Luca Passi.''
Ed è stato un quid pensare a Pinocchio, dopo aver letto le frasi di don Luca: ``Testa di legno e cuor di fuoco, amare assai e pensar poco. Procurate nel vostro operare di aver retta intenzione, e poi il Signore benedirà anche gli spropositi che poteste commettere''; e anche ``Non ti è chiesto altro che una santa amicizia'' ci ha portato a ripensare al messaggio da trasmettere grazie alle parodie delle fiabe. Col sorriso, che da sempre è il nostro biglietto da visita sul palco (facendo ``i gioppini a fin di bene''), cercando di donarlo al pubblico e come finalità degli spettacoli stessi: infatti, ogni replica è finalizzata alla raccolta fondi per donare un sorriso ai più fragili, vicini e lontani.

\figuramedia{0.9\textwidth}{pranzo.jpg}

E lo stesso sorriso, colmo di gioia, lo abbiamo ritrovato nel calore e nell’accoglienza ricevuti a Maserada: il parroco – grazie don Mirco! - e le famiglie della parrocchia ci hanno aperto le porte delle lore case e, soprattutto, il cuore per accoglierci, con quella gioia spontanea e genuina che sono merce rara al giorno d’oggi! Grazie Luca, Emma, Gilda, Elvira e Alana! Se col nostro piccolo spettacolo abbiamo cercato di trasmettere l’entusiasmo del fare del bene con gioia, questo si è moltiplicato grazie ai momenti condivisi di incontro e confronto che veramente ci hanno arricchito tantissimo, dentro.
Grazie di cuore a suor Annapaola per \textit{tutto}, a tutte le consorelle. Serberemo un ricordo speciale della Santa  Messa mattutina della domenica, tutti insieme nella cappella del convento: noi animando la liturgia coi canti al suono della chitarra, i cooperatori, i volontari e le suore, quando l’unione dei cuori è risuonato nell’unione delle voci dell’Alleluja pasquale. Dopo l’esperienza dell’ottobre scorso, quando proponemmo lo stesso spettacolo a Padova e a Brescia, non pensavamo di poter bissare le emozioni!...

Ebbene, se possibile, a Maserada la gioia di sentirsi accolti, come ``qualcuno di famiglia'' è raddoppiata! E, siccome ``Noi recitiamo solo se possiamo raccogliere offerte da donare a chi ha bisogno'' le offerte generose raccolte a teatro per l’opera delle suore dorotee in Brasile ci hanno ricordato che l’unione di intenti, porta molto frutto.

\figuramedia{0.9\textwidth}{gruppo.jpg}

Siamo una goccia (sorridente e ironica, vero!) nel mare dell’amore. Ma se quella goccia mancasse, all’oceano mancherebbe.


\firma{Stefania P. \\ Responsabile del Gruppo Recital Schignano}

Un grazie anche dal Brasile, dove è arrivato il ricavato della recita, così potranno iniziare i lavori per il bagno dei disabili. Opera necessaria e urgente per il buon funzionamento del centro.



Miei cari Compaesani,

Mi dirigo a voi, per questo mezzo,  con molta gratitudine, per la vostra generosa offerta di 1 050 Euro, ricavati nell`attività che avete avuto il 13 aprile e che ho ricevuto integralmente  in maggio.

Sono sempre tanto riconoscente a ciascuno di voi, anche a nome dei beneficiati, e prego il Signore che vi benedica e vi ricompensi abbondantemente!

Già sapete che la somma servirà per la costruzione di un bagno con accessibilità per persone disabili, in accordo a norme vigenti, nel Centro di Convivenza “Santa Dorotea”; che ospita ogni giorno circa 200 ragazzi, di condizioni precarie. E i fine settimana, per gruppi e incontri di persone del settore: perifería estremo sud di San Paolo.

I lavori cominceranno in luglio, se Dio vuole. Continuiamo, carissimi amici, ad essere dono di Dio per gli altri, dando il meglio di noi stessi, per costruire, un mondo di pace, di giustizia, di amore e fraternità.

Vi ricordo nella preghiera, soprattutto in questo tempo che celebrate la Festa della Madonna delle Vittorie e voi, ogni tanto, fatelo anche per me.

Grazie infinite e saluti cari a tutti!                                                             

\firma{Suor Loredana}
