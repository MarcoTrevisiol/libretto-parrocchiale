\section{Discorso di benvenuto a don Federico da parte del Consiglio Pastorale Parrocchiale}

Benvenuto don Federico,
a nome del consiglio pastorale e di tutta la comunità di Maserada La accogliamo con gioia nella nostra parrocchia.

Un grazie a Lei Vescovo Michele per la sua preziosa presenza che ci fa sentir ancora di più lo spirito della nostra chiesa diocesana, e grazie per aver donato una nuova guida alla nostra comunità, in una prospettiva sempre più marcata di crescita e di unione anche con le parrocchie di Varago e Candelù, e quindi di tutta la nostra collaborazione pastorale.

Un pensiero di ringraziamento, oggi, lo vogliamo rivolgere di cuore anche a don Mirco per quanto ci ha donato nei 23 anni trascorsi con noi, per essere stato un umile guida nella fede con la sua presenza generosa e costante.

Accogliamo con gioia le comunità di Candelù e Varago, che hanno desiderato accompagnare il loro pastore in questo nuovo inizio di cammino, nella consapevolezza di quanta ricchezza si vive con la collaborazione e l'unione di tradizioni spirituali vissute e consolidate. Ci auguriamo di proseguire insieme il cammino di fede in serenità, pace e gioia.

Per Lei don Federico ecco una parrocchia che da qualche settimana, in modo particolare, ha iniziato a conoscere. Ci sono case a cui bussare, fedeli da conoscere, persone che hanno bisogno di una scintilla nuova per continuare ad alimentare la fiamma della fede.

È una comunità variegata, come tanti colori che insieme formano un bellissimo quadro. Dai bambini della scuola dell'infanzia, a quelli del catechismo e scout, ai ragazzi, agli animatori del Grest, le giovani famiglie, i disabili, adulti e anziani.

Ognuno di loro ha un mondo vissuto, ragazzi che aspettano testimoni credibili e sinceri, giovani fragili, insicuri e spaesati da un futuro incerto, genitori preoccupati, famiglie bisognose, disabili e malati che cercano conforto, oltre ad altre difficili realtà. Abbiamo bisogno di parole che parlino al nostro tempo, che ci aiutino a guardare la realtà di tutti i giorni con speranza e con uno sguardo più aperto e generoso verso gli altri, come ci insegna il messaggio evangelico.

Ci sono tanti volontari che animano la nostra comunità, che dedicano il loro tempo ai ragazzi nel catechismo e nell'animazione dell'oratorio, i cori che animano le messe, gli operatori della Caritas, i sacristi, i lettori e chierichetti, i ministri straordinari, e quanti collaborano quotidianamente per il bene della parrocchia; come le cooperatrici dell'opera di S. Dorotea, e le nostre suore dorotee che da 173 anni sono tra noi, prima con il servizio, ora solo come cuore pulsante della parrocchia con la preghiera, l'offerta e l'accoglienza.

Ci sono tradizioni belle e importanti che la nostra piccola chiesa celebra con fede e devozione: la Madonna delle Vittorie, San Giorgio nostro patrono, San Rocco e San Francesco, a cui affidiamo sempre le nostre famiglie, la comunità e in modo particolare - ora - il suo cammino tra noi.

Tutti abbiamo bisogno del pastore che ci parli di Dio, abbiamo bisogno di scoprirci fratelli e compagni di viaggio, e vedere nel volto del compagno il volto di Gesù che spezza il pane con noi.

La sua presenza è preziosa e indispensabile, ci aiuta a riconoscerci discepoli e a mettere insieme le nostre sensibilità e diversità. Ma, soprattutto, ci permette di fare esperienza concreta della misericordia di Dio, che ci illumina con la sua Parola, ci guarisce e nutre con i Sacramenti.

Noi Le assicuriamo la nostra buona volontà e collaborazione, Le offriremo le nostre idee e le nostre tradizioni di popolo cristiano, assieme alla preghiera perché la sua missione sia sempre sostenuta dalla Grazia.

La nostra comunità Le fa dono di questo camice bianco, veste bianca con cui celebrerà l'eucaristia, ma anche i vari sacramenti, che ci aiuteranno, come fedeli, ad avere noi stessi quella veste bianca dell'anima che, dal giorno del nostro battesimo, siamo chiamati a indossare sempre per partecipare alla mensa del Signore.

Benvenuto tra noi, tanti auguri per il suo ministero pastorale e\dots
Buon cammino don Federico!

\firma{Lucia Polo}
