\section{Preghiera del Giubileo}

Nelle nostre comunità siamo stati invitati nel nostro percorso quaresimale a essere pellegrini di speranza. Ogni domenica siamo stati invitati a compiere un passo dopo l'altro che ci ha portato <<verso Gerusalemme>>, verso cioè il mistero pasquale dove la speranza si confronta con il suo contrario, la morte, e, per la forza dell'amore di Dio, la attraversa fino a pienezza di vita risorta. S. Paolo esprime tutto ciò così in un passo della lettera ai Romani: <<La speranza poi non delude, perché l'amore di Dio è stato riversato nei nostri cuori per mezzo dello Spirito Santo che ci è stato dato>> (Rm 5,1--2.5).

\begin{verse}
Padre che sei nei cieli, \\
la fede che ci hai donato nel \\
tuo figlio Gesù Cristo, nostro fratello, \\
e la fiamma di carità \\
effusa nei nostri cuori dallo Spirito Santo, \\
ridestino in noi, la beata speranza \\
per l'avvento del tuo Regno.


La tua grazia ci trasformi \\
in coltivatori operosi dei semi evangelici \\
che lievitino l'umanità e il cosmo, \\
nell'attesa fiduciosa \\
dei cieli nuovi e della terra nuova, \\
quando vinte le potenze del Male, \\
si manifesterà per sempre la tua gloria.


La grazia del Giubileo \\
ravvivi in noi Pellegrini di Speranza, \\
l'anelito verso i beni celesti \\
e riversi sul mondo intero \\
la gioia e la pace \\
del nostro Redentore. \\
A te Dio benedetto in eterno \\
sia lode e gloria nei secoli.

Amen
\end{verse}
\nopagebreak{\raggedleft\textit{Francesco}\par}

