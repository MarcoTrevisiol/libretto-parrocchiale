\section{L'alfabeto del Giubileo}

\figurawrapdx{0.25\textwidth}{giubileo.jpg}
\paragraph{}
\vspace*{-\parskip}

\begin{description}

  \item[N di Novità] La novità cristiana non si riduce a un messaggio teorico, ma è qualcosa che agisce, che cambia la realtà. Non è un'illusione, ma una forza che trasforma l'uomo e il mondo, rinnovandoli dall'interno, se si sceglie di rimanere ancorati a Gesù: <<se uno è in Cristo, è una creatura nuova; le cose vecchie sono passate, ecco ne sono nate di nuove.>> (2 Cor 5,17)
\end{description}

\begin{description}

  \item[O di Onnipotente] Dio è Padre onnipotente. La sua paternità e la sua potenza si illuminano a vicenda. Infatti, egli mostra la sua onnipotenza paterna attraverso il modo con cui si prende cura dei nostri bisogni; attraverso l'adozione filiale che ci dona (<<Sarò per voi come un padre, e voi mi sarete come figli e figlie, dice il Signore onnipotente>>); infine attraverso la sua infinita misericordia, dal momento che egli manifesta al massimo grado la sua potenza perdonando liberamente i peccati.

  \item[P di Porta Santa] La Porta Santa è molto più di un elemento architettonico: attraversarla significa riconoscere il bisogno di misericordia, accogliere il perdono e lasciarsi trasformare dalla grazia. Gesù stesso si presenta come porta della salvezza: <<Io sono la porta: se uno entra attraverso di me, sarà salvato>> (Gv 10,9). C'è dunque Cristo al centro del cammino giubilare: egli è la via per entrare nella comunione con Dio, nella vita nuova che non delude.

  \item[Q di Quotidianità] Papa Francesco nel Messaggio per la Giornata Mondiale della Gioventù 2023 così ha scritto: <<La speranza è il sale della quotidianità. È la certezza, radicata nell'amore e nella fede, che Dio non ci lascia mai soli e mantiene la sua promessa.>> Ci aiuti a riempire la nostra quotidianità, le nostre parole e le nostre azioni, il gusto della Speranza che nasce dalla Pasqua, certi che, nella Risurrezione, Dio ha mantenuto la sua promessa di eternità e Cristo, nostra Speranza, è risorto!

  \item[R di Regno di Dio] Il Regno di Dio è la presenza viva e operante di Dio nel mondo, che trasforma i cuori e costruisce la fraternità. È già in mezzo a noi e tutti siamo chiamati a contribuire: <<Il Regno di Dio cresce attraverso il lievito nascosto della speranza e del servizio silenzioso>> (Spes non confundit).
\end{description}

Continua \ldots

