\section{L'alfabeto del Giubileo}
\begin{center}
\textit{da un'idea dell'unità pastorale di Pieve di Soligo}
\end{center}

\begin{description}
  \item[N di Novità] La novità cristiana non si riduce a un messaggio teorico, ma è qualcosa che agisce, che cambia la realtà. Non è un'illusione, ma una forza che trasforma l'uomo e il mondo, rinnovandoli dall'interno, se si sceglie di rimanere ancorati a Gesù: <<se uno è in Cristo, è una creatura nuova; le cose vecchie sono passate, ecco ne sono nate di nuove.>> (2 Cor 5,17)

  \item[O di Onnipotente] Dio è Padre onnipotente. La sua paternità e la sua potenza si illuminano a vicenda. Infatti, egli mostra la sua onnipotenza paterna attraverso il modo con cui si prende cura dei nostri bisogni; attraverso l'adozione filiale che ci dona (<<Sarò per voi come un padre, e voi mi sarete come figli e figlie, dice il Signore onnipotente>>); infine attraverso la sua infinita misericordia, dal momento che egli manifesta al massimo grado la sua potenza perdonando liberamente i peccati.

  \item[P di Porta Santa] La Porta Santa è molto più di un elemento architettonico: attraversarla significa riconoscere il bisogno di misericordia, accogliere il perdono e lasciarsi trasformare dalla grazia. Gesù stesso si presenta come porta della salvezza: <<Io sono la porta: se uno entra attraverso di me, sarà salvato>> (Gv 10,9). C'è dunque Cristo al centro del cammino giubilare: egli è la via per entrare nella comunione con Dio, nella vita nuova che non delude.

  \item[Q di Quotidianità] Papa Francesco nel Messaggio per la Giornata Mondiale della Gioventù 2023 così ha scritto: <<La speranza è il sale della quotidianità. È la certezza, radicata nell'amore e nella fede, che Dio non ci lascia mai soli e mantiene la sua promessa.>> Ci aiuti a riempire la nostra quotidianità, le nostre parole e le nostre azioni, il gusto della Speranza che nasce dalla Pasqua, certi che, nella Risurrezione, Dio ha mantenuto la sua promessa di eternità e Cristo, nostra Speranza, è risorto!

  \item[R di Regno di Dio] Il Regno di Dio è la presenza viva e operante di Dio nel mondo, che trasforma i cuori e costruisce la fraternità. È già in mezzo a noi e tutti siamo chiamati a contribuire: <<Il Regno di Dio cresce attraverso il lievito nascosto della speranza e del servizio silenzioso>> (Spes non confundit).
\end{description}

\newpage
\figurawrapdx{0.45\textwidth}{giubileo.jpg}
\paragraph{}
\vspace*{-\parskip}
\begin{description}
  \item[S di Soglia] Il Giubileo ci pone davanti a una soglia da attraversare: quella della Porta Santa, certo, ma anche quella simbolica di una vita rinnovata. Ogni soglia segna un passaggio, un \textit{prima} e un \textit{dopo}, un invito a lasciare il vecchio e ad accogliere il nuovo. Entrare per la soglia del Giubileo significa aprirsi alla grazia, al perdono, alla speranza. Ma perché il passaggio sia autentico, serve il coraggio di cambiare, di convertirsi, di rimettere al centro Dio. In questo tempo di grazia, chiediamo al Signore di donarci occhi per riconoscere le soglie che abbiamo davanti e cuore per attraversarle con fiducia.
\end{description}

\begin{description}
  \item[T di Testimonianza] La collaborazione Pastorale di Maserada e Breda in occasione dell'anno giubilare ha organizzato un pellegrinaggio alla Basilica della Madonna dei Miracoli di Motta di Livenza la domenica del 23 marzo, evento aperto a tutte le sette comunità appartenenti alla stessa. L'organizzazione dell'evento, prevedeva per chi voleva il trasporto in pullman fino a Chiarano, che dista circa sette chilometri dal Santuario.

  A piedi poi il gruppo di pellegrini ha proseguito con due tappe intermedie, durante le quali sono stati presentati due segni di \textit{Speranza} attraverso due testimonianze. La prima di un giovane che ha illustrato il suo progressivo avvicinamento alla fede, la seconda di un ex detenuto, che ha ripercorso le tappe del proprio cammino, fino alla ripresa dei rapporti con la società e la famiglia. Due esperienze davvero forti che dimostrano come la Speranza sia presente ancora nel cuore delle persone.

  La nostra presenza è stata molto apprezzata dai Padri Francescani, che si sono messi a disposizione per le confessioni. La Santa Messa ha poi concluso il pellegrinaggio lasciando nel cuore dei fedeli una grande gioia. Un particolare e doveroso ringraziamento a Don Federico, don Mario e don Giorgio per le testimonianze di fede raccontate e per l'accompagnamento in preghiera durante il cammino verso la Basilica.

  Un gruppo di \textit{\sout{partecipanti} testimoni}
\end{description}

Continua \ldots

