\section{Auguri di don Federico}

Cari parrocchiani in questo tempo di Pasqua in questo anno giubilare il mio augurio è che possiamo riconoscere e accogliere l'amore di Dio che è stato riversato nei nostri cuori. In un contesto dove è più facile vedere quello che non va, in un contesto di paura e incertezza per il futuro, in forza dell'amore di Dio che è stato riversato nei nostri cuori vi invito a saper riconoscere i segni di speranza.

Condivido con voi alcuni segni di speranza sperimentati in questo tempo: mi ha colpito accompagnando una coppia al matrimonio, come questi giovani hanno riconosciuto nell'arco della loro vita dentro gli eventi lieti e tristi, o di prova essere stati accompagnati dal Signore, come hanno sperimentato la sua presenza amorevole e discreta.

Ne riporto un altro con un messaggio di una mamma dopo l'uscita genitori dei cresimandi: <<Grazie per la giornata di ieri che non è stata solo importante per i nostri figli, ma anche per noi genitori. Io, in particolar modo, ho finalmente compreso il valore del sacramento della Santa Cresima, quanto sia bello e importante, il suo reale significato\ldots Grazie per le risate, per la simpatia, per averci fatto sentire a nostro agio\ldots>>

Poi la celebrazione della prima confessione dei bambini di tutte e tre le parrocchie, momento di festa per accogliere la gratuità del perdono di Dio e di comunità insieme che hanno condiviso la preparazione con le catechiste e il momento conviviale insieme ai genitori.

Anche il pellegrinaggio della Collaborazione del 23 marzo alla Madonna di Motta. Nel percorso a piedi accogliendo l'invito a mettersi in cammino con le testimonianza della conversione di un giovane e di una persona che ha finito l'esperienza del carcere e che ha goduto dell'accoglienza del Sicomoro a Varago. Concludendo con la celebrazione eucaristica con la presenza di tutti i pellegrini.

Nella vista agli anziani e ammalati vedere la cura la dedizione dei famigliari per i loro cari, assaporare la fede semplice e genuina anche in chi vive il tempo dell'anzianità e della malattia.

Questi sono alcuni segni di speranza, la luce del risorto illumini i nostri cuori per saper scorgere i tanti segni dell'amore di Dio riversato nei nostri cuori ed essere così pellegrini di speranza.

Buona Pasqua!

\firma{Il vostro parroco --- Don Federico}

\clearpage

Nelle nostre comunità siamo stati invitati nel nostro percorso quaresimale a essere pellegrini di speranza. Ogni domenica siamo stati invitati a compiere un passo dopo l'altro che ci ha portato <<verso Gerusalemme>>, verso cioè il mistero pasquale dove la speranza si confronta con il suo contrario, la morte, e, per la forza dell'amore di Dio, la attraversa fino a pienezza di vita risorta. S. Paolo esprime tutto ciò così in un passo della lettera ai Romani: <<La speranza poi non delude, perché l'amore di Dio è stato riversato nei nostri cuori per mezzo dello Spirito Santo che ci è stato dato>> (Rm 5,1--2.5).
