\section{Auguri di don Federico}

Cari parrocchiani,
In questo tempo estivo celebriamo la sagra ricordando la devozione alla Madonna delle Vittorie. In questa occasione mi viene spontaneo riandare al 20 ottobre scorso giorno del mio ingresso da parroco a Maserada, accompagnato dai parrocchiani di Candelù e Varago, prima di arrivare in chiesa, ho fatto tappa proprio nella chiesetta, accolto dai bambini della scuola materna e da tanta altra gente. Ho voluto affidare questo nuovo servizio a Maria lei che è madre nostra possa intercedere per tutti noi, per il nostro cammino di collaborazione, per gli ammalati per le nostre necessità.

Festeggiare Maria diventa per la nostra comunità occasione per fermarci, prendere coscienza della nostra umanità, delle nostre e altrui fragilità e del contesto di incertezza, di violenza nel quale ci troviamo a vivere. E se questo può generare sfiducia, affidarci alla sua materna intercessione diventa fonte di rinnovata speranza, soprattutto perché Maria ci indica Gesù e guardando a lui siamo chiamati ancora una volta a scegliere di credere che seguirlo porti oltre la crepa della morte, al di là dell'abisso, nella vita strabordante di Risurrezione\ldots

Ma scegliere di credere comporta insieme sempre anche certamente rimanere con lui sulla croce, condividere lo strazio di chi è calpestato, contribuire a portare la croce di chi è abbandonato, mettere in comune le molte o poche risorse di ciascuno ciascuna di noi con chi ha perduto tutto\ldots agire per la giustizia, per la custodia del nostro essere umani, per l'insopprimibile sete di vita che grida dal cuore di ogni uomo e donna, di ogni vivente.

Sull'esempio di Maria confidando in lei chiediamo la grazie di scegliere di credere per essere dentro la comunità e nel mondo segno tangibile dell'amore di Dio che genera rinnovata speranza.

Sia anche questo tempo estivo opportunità di riposo, di rigenerarsi nel corpo e nello spirito.


\firma{Il vostro parroco \\ Don Federico}
