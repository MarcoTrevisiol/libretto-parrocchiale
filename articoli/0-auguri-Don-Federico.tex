\section{Il nostro sguardo da bambino}

Cari parrocchiani,

in occasione del festa del Natale mi rivolgo a voi da partire un passo del Vangelo: <<Io ti rendo lode, Padre, Signore del cielo e della terra, che hai nascosto queste cose ai dotti e ai sapienti e le hai rivelate ai piccoli>> (Lc 10,21).

Gesù ci ricorda che per vivere la vera gioia del vangelo, ma anche del suo Natale, dobbiamo assumere lo sguardo del bambino, che non è uno sguardo perso o poco attento, nemmeno uno sguardo superficiale; è lo sguardo di chi sa cogliere la bellezza nelle piccole cose, di chi sa meravigliarsi anche per ciò che ai nostri occhi appare scontato e di chi desidera conoscere l'Altro senza timore.

In questo anno giubilare che ormai volge al termine il mio augurio è quello di aver fiducia di riscoprire <<il nostro sguardo da bambino>>. Uno sguardo attento e profondo che sa riconoscere l'Emmanuele, il Dio con noi, in ogni gesto di cura e di dignità verso la carne umana, specialmente quella più povera e bisognosa, quelle che non attira l'attenzione. Uno sguardo che non si ferma all'apparenza delle luci, ma che sa andare in profondità per ascoltare con attenzione chi ci vive accanto. Uno sguardo che sa sorprendersi della bellezza della fedeltà di Dio dentro le diverse situazioni della vita, per sperimentare che non siamo mai soli e diventare così pellegrini di speranza. Dio è il primo pellegrino che ci è venuto incontro facendosi uomo in Gesù, in una umanità fragile, affinché noi accogliendolo, possiamo rinascere in Lui come figli di Dio e della pace.

Questa luce che ci viene donata possa aiutarci nel nostro cammino, dentro le situazioni lieti e tristi o di prova della vita.

Rinnovo il mio ringraziamento a Dio per quanti nella comunità si adoperano nel svolgere un servizio: i membri degli organismi pastorali, coloro che si dedicano per la cura degli ambienti, della chiesa, della canonica, oratorio e di altre iniziative importanti per sostenere la parrocchia. E a chi nel nascondimento con umiltà offre la sua preghiera per il bene della parrocchia. Rinnovo il mio ricordo per tutta la comunità e i miei auguri di Buon Natale e di un nuovo anno di vita e di pace.


\firma{Il vostro parroco \\ Don Federico}
