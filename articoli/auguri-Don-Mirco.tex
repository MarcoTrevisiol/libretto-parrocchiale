\section{Gli auguri di Don Mirco}
Cari parrocchiani,

voglio raggiungervi con questo scritto per porgervi i miei auguri pasquali. La primavera è arrivata , dopo un inverno piuttosto mite, mostrando la bellezza dei suoi fiori che rallegrano il nostro animo e creano il giusto sfondo per la festa di Pasqua. Essa ci porta la gioia di una vita nuova che è sbocciata, e insieme è un simbolo della rinascita offerta da Gesù risorto.

Il suo amore che l’ha portato a donare tutto se stesso per noi, con questa festa ci sprona a essere portatori di speranza per una nuova umanità. Sempre ci stupisce il fatto sorprendente, che Gesù è risuscitato dai morti, è ritornato vivo in mezzo ai suoi e li ha ricolmati di fiducia e di letizia. Questo evento stupendo, è stato proclamato subito dagli apostoli, che hanno esortato tutti ad accogliere il Cristo come il nostro unico salvatore e Signore.

La madre Chiesa, celebra ogni anno il triduo pasquale, iniziando dal giovedì santo, giorno della sua ultima cena con i suoi amici. Il venerdì santo medita la sua passione dolorosa fino alla morte in croce. Il sabato santo giorno del lutto, attende in preghiera silenziosa fino alla sera, quando con la Veglia pasquale, rivive la Resurrezione di Cristo. Partecipiamo a questi momenti di spiritualità, che ci donano la gioia di incontrare Gesù, il nostro buon pastore che ha dato la sua vita per noi.

L’incontro con lui, attraverso le celebrazioni liturgiche in chiesa, vi aiuterà a vivere una bella festa Pasquale, apportatrice di speranza nella vostra vita, di nuovo impegno per costruire la pace, la giustizia e l’armonia fra di noi.

Per migliore accoglienza della grazia del Signore, accostiamoci al sacramento del suo perdono, che ci libera dai peccati commessi e ci immette nel cuore un’energia divina per risorgere come credenti. Riconciliamoci anche fra noi e allora gusteremo una autentica gioia pasquale.

A voi e alle vostre famiglie auguro una santa Pasqua nel Signore.

\firma{Il vostro parroco Don Mirco}
