\section{Teatro Don Luca Passi}


Era l’anno 1852 quando per le strade di Maserada si aggirava un sacerdote accompagnato da cinque suore, le vesti lunghe e nere di entrambi attiravano l’attenzione della gente intenta ai loro lavori quotidiani.

I bambini con curiosità si avvicinavano e il sorriso di quel sacerdote e lo sguardo materno delle suore li incoraggiava a farsi vicini. I più coraggiosi fanno qualche domanda.

``Da dove venite?... Perché siete qui?... ''

La risposta non si fa attendere. ``Veniamo da Venezia e siamo qui per fare amicizia con voi e per aiutarvi a diventare sempre più amici di Gesù.

``Io -- dice il Sacerdote -- mi chiamo don Luca Passi, e siccome amo tanto la gioventù ho voluto che queste suore mi aiutassero per tessere una rete di amicizia con le mamme, i papà, con le persone che amano la gioventù, per far in modo che, bambini, ragazzi e giovani abbiano vicino compagni di viaggio che li guidano e li sostengono.  Io questa sera tornerò a Venezia per poi andare ad annunciare l’amore di Gesù in altri paesi e a incontrare altre persone, loro rimarranno  qui con voi.''

Da allora la presenza è stata quasi continua. L’assenza da Maserada fu causata solo da eventi bellici, prima nel 1866 e poi nel 1917.

Ci sembrava perciò bello, anche su suggerimento di don Mirco, far conoscere un po’ quel sacerdote che  oltre 170 anni fa è venuto qui, per offrire anche agli abitanti di Maserada quel dono di Grazia che lo Spirito gli ha dato, per questo sabato 13 aprile 2024 sarà offerto un recital da un gruppo teatrale di Como che leggendo alcuni scritti del Beato Luca Passi, hanno cercato di interpretare in modo originale il suo messaggio.

Lo spettacolo è gratuito, però ci sembrava bello che ci fosse un segno di solidarietà verso i più poveri.

Abbiamo quindi pensato di raccogliere delle offerte in favore della Missione  del Brasile dove lavora la nostra sorella Maseradese suor Loredana Celebrin.

Ci auguriamo una numerosa partecipazione.

\firma{Suor Annapaola}

\figuragrande{locandina-teatro.jpg}
