\section{Gli auguri di Don Mirco}
Cari parrocchiani,

sono giunti i giorni dedicati alla sagra della Madonna delle Vittorie. Come passa presto il tempo e quante vicende abbiamo vissuto sia come singoli sia come comunità civile e religiosa. Alcuni di noi hanno vissuto momenti di gioia, altri di tristezza, di preoccupazione, e di dolore. Diverse persone hanno lasciato questa terra per ritornare alla casa del Padre, ma pochi sono nati per colmare il vuoto rimasto. Di fronte alla crisi famigliare, economica, morale, sociale e religiosa, siamo tentati a ripiegarci in noi stessi nel nostro piccolo mondo. È questa una tentazione che è figlia dell’individualismo molto diffuso e della conseguente indifferenza nella nostra società. Anche la partecipazione attiva come credenti e come cittadini è calata, forse per mancanza di fiducia nelle istituzioni, per la perdita di motivazioni morali e spirituali. E oltre a tutto questo, le guerre in atto hanno contribuito a tirarci giù di morale   e a perdere la speranza e la fiducia in un mondo più sicuro e più fraterno.

Ma allora che senso ha celebrare questa festa in onore di Maria?  È ringraziare e gioire per  il suo amore fedele, riversato su di noi  discepoli del suo Figlio. È riconoscere  che da tanti anni ci viene incontro per sostenerci nella lotta contro il maligno, che vuole distruggerci allontanandoci dal Signore con tutti i mezzi. Nei suoi messaggi ci ricorda che non c’è futuro positivo senza Dio, non c’è vera pace senza di Lui, non c’è giustizia  al di fuori dei comandamenti che ci ha dato. Pertanto noi onoriamo  Maria santissima, perché ci insegna la via della serenità, dell’impegno, della speranza, della pace nelle nostre famiglie, il sentiero per andare in Cielo.  Come madre buona che ha già raggiunto la beatitudine eterna, ci insegna con amore la strada della felicità e della pace.

Auguro davvero a  tutti di accostarvi a lei in questi giorni per offrire le vostre preghiere e i vostri propositi di una fede cristiana più convinta e partecipe ai sacramenti che la Chiesa offre per il bene di ciascuno di voi.

Buona festa a tutti con Maria.

\firma{Il vostro parroco Don Mirco}
