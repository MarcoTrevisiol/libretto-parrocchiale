\section{Caritas di Candelù, Maserada e Varago}

Gli ambiti della nostra Caritas Interparrocchiale nei quali operiamo come Volontari Gruppo Caritas sono vari: Accoglienza e Fraternizzazione con persone che hanno bisogno del nostro aiuto personale o tramite il Centro di Ascolto -- preparazione e distribuzione borse spesa -- raccolta e distribuzione vestiario o suppellettili per la casa -- ritiro e consegna mobili o elettrodomestici -- aiuti economici per pagamenti vari nei momenti di difficoltà economiche dovute alla mancanza di lavoro.

Tutto ciò l’abbiamo già potuto leggere negli articoli dei precedenti libretti Parrocchiali degli scorsi anni.
L’informazione che leggerete in questo libretto “Pasqua 2024” è puramente tecnica ma significativa in quanto è coinvolta tutta la nostra comunità e riguarda le borse spesa, che ogni 15 giorni vengono ritirate dai nuclei familiari. Le borse spesa sono composte da viveri di prima necessità come pasta, latte, legumi, passate di pomodoro, zucchero, olio di semi e oliva, farina o da altri alimenti che possono essere biscotti, scatolami di tonno, marmellate e prodotti per l’igiene personale o pulizia della casa.

\subsection{Resoconto relativo all’anno 2023}
Sono stati 27 i casi in carico nel corso del 2023 così suddivisi:
\begin{itemize}
	\item 16 nuclei italiani, per un totale di 26 persone di cui 24 adulti e 2 minori;
	\item 11 nuclei stranieri, per un totale di 44 persone di cui 23 adulti e 21 minori.
\end{itemize}

I prodotti delle borse spesa variano a seconda della composizione dei nuclei familiari. Il numero delle borse spesa consegnate nell’anno 2023 sono i seguenti:
\begin{description}
	\item[Tipo A] per nuclei familiari composti da 1-2 persone, n. 292;
	\item[Tipo B] per nuclei familiari composti da 3 o più persone, n. 198.
\end{description}

Vi assicuriamo che il quantitativo di generi alimentari e prodotti vari che viene distribuito durante l’anno ai nuclei familiari è notevole.

Dobbiamo anzitutto ringraziare l’Amministrazione Comunale e vari privati cittadini che con i loro contributi e offerte ci permettono di acquistare parte prodotti alimentari, alcuni tipi di alimenti ci pervengono dall’ AGEA e dalla FEAD, altri ancora vengono raccolti nella giornata della Colletta Solidale presso i vari negozi o supermercati del Comune. A tal proposito esprimiamo un grazie sentito a tutti i cittadini che generosamente donano e a tutti volontari che con lodevole impegno ci hanno aiutato nella raccolta.

\firma{A tutti auguriamo di cuore una serena Pasqua di Resurrezione\\
Volontari Gruppo Caritas Interparrocchiale}

