\section{Caritas di Candelù, Maserada e Varago}
Gli ambiti della nostra Caritas Interparrocchiale nei quali operiamo come Volontari Gruppo Caritas sono vari: Accoglienza e Fraternizzazione con persone che hanno bisogno del nostro aiuto personale o tramite il Centro di Ascolto - preparazione e distribuzione borse spesa - raccolta e distribuzione vestiario o suppellettili per la casa - ritiro e consegna mobili o elettrodomestici - aiuti economici per pagamenti vari nei momenti di difficolt� economiche dovute alla mancanza di lavoro.                         
Tutto ci� l�abbiamo gi� potuto leggere negli articoli dei precedenti libretti Parrocchiali degli scorsi anni.
L�informazione che leggerete in questo libretto �PASQUA 2024� � puramente tecnica ma significativa in quanto � coinvolta  tutta la nostra comunit� e riguarda le borse spesa,che ogni 15 giorni vengono ritirate dai nuclei famigliari.
Le borse spesa sono composte da viveri di prima necessit� come pasta, latte, legumi, passate di pomodoro, zucchero, olio di semi e oliva, farina o da altri alimenti che possono essere biscotti, scatolami di tonno, marmellate e prodotti per l�igiene personale o pulizia della casa.
\textbf{\underline{Resoconto relativo all’anno  2023}}
Sono stati 27 i casi in carico nel corso del 2023 cos� suddivisi:
- Nr. 16 Nuclei Italiani:
-Tot. Nr  26 persone di cui Adulti Nr. 24  e Minori Nr. 2
-Nr. 11 Nuclei Stranieri:
-Tot. Nr. 44 persone di cui Adulti Nr. 23 e  Minori Nr. 21

I prodotti delle borse spesa variano a seconda della composizione dei nuclei famigliari.
Il numero delle borse spesa consegnate nell�anno 2023 sono i seguenti:
\textbf{Tipo A} - Per nuclei famigliari composti da 1-2 persone nr. 292
\textbf{Tipo B} - Per nuclei famigliari composti da 3 o pi� persone nr.198
Vi assicuriamo che il quantitativo di generi alimentari e prodotti vari che viene distribuito durante l�anno ai nuclei famigliari � notevole.
Dobbiamo anzitutto ringraziare l�Amministrazione Comunale e vari privati cittadini che con i loro contributi e offerte ci permettono di acquistare parte prodotti alimentari, alcuni tipi di alimenti ci pervengono dall� AGEA e dalla FEAD, altri ancora vengono raccolti nella giornata della Colletta Solidale presso i vari negozi o supermercati del Comune. A tal proposito esprimiamo un grazie sentito a tutti i cittadini che generosamente donano   e a tutti volontari che con lodevole impegno ci hanno aiutato nella raccolta.

A TUTTI AUGURIAMO di CUORE UNA SERENA PASQUA di RESURREZIONE
VOLONTARI GRUPPO CARITAS INTERPARROCCHIALE

\firma{Il gruppo volontari Caritas}
