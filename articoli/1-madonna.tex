\section{Madonna delle Vittorie}

La leggenda popolare attribuisce all'icona che si trova nel tempietto dedicato alla Madonna delle Vittorie, alcuni miracoli in seguito all'apparizione della Vergine ad una fanciulla chiamata Zanetta Bariviera il 9 luglio del 1722
\figurawrap{0.55\textwidth}{doge}

Non lontano dalla parrocchia, seguendo viale Caccianiga, troviamo il piccolo tempio dedicato alla Madonna delle Vittorie. Fu eretto nel 1717 dal doge veneziano Carlo Ruzzini proprio accanto alla sua villa in campagna.
\figureafianco{tempietto}{processione}

All'interno vi è un'immagine, che è una copia di quella della chiesa di Santo Stefano a Vienna, che presenta la Vergine con il Bambino.

La leggenda popolare attribuisce a questa icona miracoli dopo l'apparizione della Madonna ad una fanciulla chiamato Zanetta Bariviera il 9 luglio del 1722. La data è ricordata oggi con solenni cerimonie.

Nel 1917, secondo centenario della fondazione del tempietto, il vescovo di Treviso monsignor Longhin, incoronava solennemente l'immagine che, dopo le giornate di Caporetto, venne portata in diverse località del Veneto. Essa ritornò in paese il 6 luglio del 1919, trainata da quattro cavalli, con i trofei della vittoria.

\firma{Fr. John Francesco Maria Lim}
