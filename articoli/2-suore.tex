\section{Anche noi <<pellegrine di speranza>>}

Come sempre nei tempi forti, nella nostra cappella sono presenti i segni che tracciano il percorso proposto dalla Collaborazione, questo perché ci sentiamo parte viva della comunità parrocchiale e, poiché le forze fisiche non permettono, alla maggior parte di noi, di essere presenti fisicamente alle attività della parrocchia, questo ci sembra un modo per esserci. Non possiamo dimenticare quello che il nostro Fondatore, il Beato Luca Passi, diceva alle sue prime suore: <<La chiesa parrocchiale, sia la vostra cappella, le strade della parrocchia, siano il vostro chiostro.>>

\figuramedia{0.9\textwidth}{suore.jpg}

Il fatto poi, che la nostra cappella sia diventata, in questo periodo, quasi la chiesa parrocchiale feriale, ci fa molto piacere, perché ci permette di essere aperte al territorio, questo ci chiede il nostro DNA di <<Dorotee>>.

Per questa Quaresima il percorso proposto ci apre in modo efficace al cammino giubilare. I segni ci riportano al pellegrinaggio. Ci indicano tutto ciò che il pellegrino deve avere per raggiungere con successo la meta desiderata. Oltre \textbf{agli scarponi}, \textbf{lo zaino} e \textbf{il bastone}, equipaggiamento adeguato per partire, abbiamo visto \textbf{la bussola}, necessaria per indicare la rotta che vogliamo prendere (andare verso Dio), \textbf{la mappa}, per non prendere strade sbagliate, (il Vangelo), \textbf{la borraccia}, per dissetarci e dissetare, (Incontrare Cristo, fonte d'acqua viva) \textbf{la coperta}, perché il freddo non ci fermi (lasciarci scaldare dal fuoco dell'Amore divino) \textbf{la lanterna}, perché il buio non ostacoli il cammino (ricercare Cristo vera luce), \textbf{la pagnotta}, per sostenerci nelle fatiche della viaggio (lasciarci nutrire da Cristo, Pane vivo di salvezza).

La bolla di indizione scritta da Papa Francesco ha come titolo \textit{Spes non confundit} cioè \textit{La speranza non delude} un invito quindi a vivere la speranza, nel nostro tempo che ci da' spesso segni di disperazione.

Mons. Rino Fisichella, responsabile dell'organizzazione del Giubileo in una sua intervista così si esprime: <<Questo Giubileo è davvero l'occasione per rianimare nel cuore di ciascuno la passione per la Speranza che non delude. Perché le donne e gli uomini di oggi hanno bisogno di speranza adesso. Tutti possono sperare, ma è il contenuto della speranza, che qualifica l'atto e lo fa comprendere diverso dal sentimento o dall'utopia. La speranza cristiana è la certezza del compimento della promessa di Dio. La speranza, insomma, è anzitutto attesa della rivelazione piena e definitiva del Signore, che si trasforma in fiducia nella sua promessa che egli verrà. Questa richiede pazienza, per non cedere allo scoraggiamento. In Dilexit nos, Papa Francesco lega il tema della speranza a uno dei desideri più profondi dell'uomo: poter sperare che ``ogni ferita possa essere guarita, anche se profonda.'' Attraverso la grazia dell'indulgenza giubilare, ciascuno di noi potrà fare esperienza dell'amore sovrabbondante di Dio che perdona e cura le ferite dei cuori. Spero che l'Anno Santo sia un'opportunità per rafforzare la certezza che Dio ci è sempre vicino, che non ci abbandona mai.>>

Auguriamoci reciprocamente di fare un vero cammino di Resurrezione per riscoprire l'Amore gratuito di Dio, che nonostante le nostre fragilità ci ama e ci perdona sempre.

\firma{Buona Pasqua!}

