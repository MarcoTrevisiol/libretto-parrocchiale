\section{Discorso di benvenuto a Don Federico da parte del sindaco}

Maserada sul Piave, 20 ottobre 2024

Eccellenza Reverendissima Monsignor Michele Tomasi, Molto Reverendo Don Federico Giacomini

A nome mio personale e dell’intera Amministrazione Comunale, certo di interpretare i sentimenti della comunità parrocchiale maseradese, ma direi dell’intera comunità maseradese, porgo un sentito e deferente saluto di benvenuto.

A Lei, Eccellenza Reverendissima, anche un sentito ringraziamento; innanzitutto perché ci onora, quale Vescovo della Diocesi di Treviso, della sua presenza, ma soprattutto perché ci ha fatto questo grande dono di nominare, nella persona di don Federico, il nuovo Pastore di questa chiesa. È un dono graditissimo, che, ne sono certo, la comunità di Maserada non solo apprezzerà, ma custodirà gelosamente.

A Lei, don Federico, grazie di aver benevolmente accolto l’invito del Vescovo a divenire la nuova guida spirituale di questa parrocchia. La comunità che l’accoglie è una bella realtà; è una comunità attiva, viva nella fede, fiera della propria identità, generosa, solidale, che sa aprirsi agli altri, disponibile al dialogo, sicuramente pronta a proseguire, sotto la Sua guida, il cammino già avviato, in una continuità desiderosa di ulteriore crescita.

Come Sindaco, ed a nome di tutta l'Amministrazione Comunale, Le assicuro la nostra più totale apertura e disponibilità al dialogo ed alla collaborazione, così come già, del resto, avviene per quanto riguarda le comunità parrocchiali di Candelù e Varago che la vedono, ormai da quattro anni, loro Parroco.

Credo che per entrambi, pur nel rispetto dei diversi ruoli, il bene primario da perseguire sia quello dell’impegno a favore di tutta la collettività avendo come unico fine il benessere e la crescita dell’intera comunità.

Benvenuto don Federico. Intraprenda con fiducia questo nuovo cammino. Le saremo vicini. Un fraterno abbraccio nel Signore.

\firma{Lamberto Marini }
