\section{Invito alla festa del 50esimo di sacerdozio del parroco}

Quest’anno il nostro parroco don Mirco Moro, per ringraziare il Signore del 50esimo di consacrazione sacerdotale, ha espresso il desiderio di organizzare una festa. Ad essa, oltre agli amici preti, ai parenti e alle suore, possono partecipare i fedeli laici che vogliono condividere questo evento, sempre più raro in questi anni. La comunità cristiana gioisce col suo pastore che è al servizio fra noi da ormai 23 anni.

La festa sarà
{\Large domenica 22 settembre 2024}.

Il programma prevede:

Alle 10.30 la S. Messa solenne concelebrata con alcuni preti e animata dal Coro S. Giorgio, terrminata la celebrazione, ci avvia verso la scuola media, dove si svolgerà il convito, previsto per mezzogiorno.

Ci sarà un aperitivo di benvenuto davanti all’ingresso della sala da pranzo. Alle 12.30 la sistemazione di ciascuno nella sala mensa nei posti già assegnati. Dopo la benedizione si comincerà il pranzo con una pastasciutta speciale, seguirà il secondo che prevede la carne allo spiedo e contorni vari.
Verso le ore 14.15 ci sarà un break con saluti ringraziamenti e canti. Infine il dolce e il caffè.

Alla conclusione del pranzo ci avvieremo, con calma, verso la chiesa parrocchiale dove alle 16.15 ci sarà la presentazione del Concerto per Coro orchestra e organo, che si prevede duri poco più di un'ora.

Per poter organizzare questa festa, c’è bisogno di prenotarsi in anticipo e pertanto il gruppo organizzatore ha deciso di fare l’iscrizione, nei sabati e domeniche 1 -- 8 -- 15 settembre versando il contributo di € 20,00. I posti per i paesani sono limitati a 120. (Ci si può iscrivere anche in canonica al sabato mattino.)

Ringraziamo fin d’ora il gruppo Alpini e gruppo S. Francesco che si sono resi disponibili per organizzare il pranzo.


\firma{Il gruppo organizzatore assieme al parroco}
