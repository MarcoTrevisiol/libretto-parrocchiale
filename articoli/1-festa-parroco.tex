\section{Invito alla festa del 50esimo di sacerdozio del parroco}

Quest’anno il nostro parroco don Mirco Moro, per ringraziare il Signore del 50esimo di consacrazione sacerdotale, ha espresso il desiderio di organizzare una festa. Ad essa, oltre agli amici preti, ai parenti e alle suore, possono partecipare i fedeli laici che vogliono condividere questo evento, sempre più raro in questi anni. La comunità cristiana gioisce col suo pastore che è al servizio fra noi da ormai 23 anni.

Il programma della festa, la domenica 22 settembre 2024, prevede:

Alle 10.30 la S. Messa solenne concelebrata con alcuni preti e animata dal Coro S. Giorgio. Terminata la celebrazione, c’è uno scambio di saluti ci avvia verso la scuola media, dove si svolgerà il convito.

Alle 12.00 l’aperitivo, davanti all’ingresso della sala da pranzo.

Alle 12.30 la sistemazione di ciascuno nella sala mensa nei posti già assegnati.

Alle 12.45 circa la benedizione e la degustazione del primo, una pastasciutta speciale. Poi passerà il secondo che comprende la carne allo spiedo e contorni vari.

Verso le ore 14.15 ci sarà un break con saluti ringraziamenti e canti.

Alle 14.45 il dolce e caffè.

Alle 15.30 Conclusione del pranzo e avvio con calma verso la chiesa parrocchiale.

Alle 16.15 Presentazione del Concerto per Coro orchestra e organo.

Alle 16.30 Inizio del Concerto che durerà circa un’ora abbondante.

Per poter organizzare questa festa, c’è bisogno di prenotarsi in anticipo e pertanto il gruppo organizzatore ha deciso di fare l’iscrizione, nei sabati e domeniche 1 -- 8 -- 15 settembre versando il contributo di € 20,oo. I posti per i paesani sono limitati a 120. Ci si può iscrivere anche in canonica al sabato mattino.

Ringraziamo fin d’ora il gruppo Alpini e gruppo S. Francesco che si sono resi disponibili per organizzare il pranzo.

\firma{Il gruppo organizzatore assieme al parroco}
