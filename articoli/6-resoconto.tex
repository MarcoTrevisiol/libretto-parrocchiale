\section{Dal CPAE}

Nella primavera del 2024 la Parrocchia si è impegnata nell'eseguire i lavori di manutenzione della gradinata d'accesso all'Auditorium <<Don Enrico Vidotto>> con il completo rifacimento del rivestimento esistente.

L'intervento si è reso necessario in conseguenza al pessimo stato in cui versava il preesistente rivestimento con distaccamenti diffusi degli elementi in pietra e ristagni d'acqua in occasione di eventi piovosi particolarmente intensi.

Nel dettaglio, il rifacimento della gradinata è consistito nella demolizione della pavimentazione con scarifica del massetto fino a solaio, l'adeguata impermeabilizzazione del fondo e la ridefinizione del profilo della gradinata con riporto in quota in sabbia e cemento, la posa della nuova pavimentazione con finitura antiscivolo certificata previa protezione impermeabile del nuovo massetto e la posa del nuovo battiscopa.

L'intervento dovrà concludersi quest'anno con l'installazione di due nuovi corrimano in acciaio inox, a sostituzione del singolo corrimano centrale precedente, più in linea con gli ingressi e le uscite esistenti dell'edificio.

Per l'intervento di rifacimento della gradinata è stata sostenuta una spesa complessiva di € 33 000, di cui € 4 000,00 coperta da un contributo generosamente erogato da un Istituto bancario del territorio e la quota parte restante da fondi propri della Parrocchia.

In questi mesi la Parrocchia si trova ad affrontare una nuova urgenza relativa, in primis, la sostituzione del battente della campana maggiore (attualmente c'è un battacchio provvisorio) e la necessaria regolazione dell'arco di oscillazione delle 3 campane principali per poterne limitare l'usura, preservandone il suono. Risulta importante procedere anche alla chiusura del foro che mette in comunicazione il fusto con la cella campanaria, per garantirne la tenuta all'acqua ed il castello delle campane dovrebbe essere restaurato e rinforzato con ulteriori elementi metallici.

Il preventivo di spesa per questi interventi ammonta ad € 9 530 IVA inclusa che, tuttavia, data la disponibilità economica derivante dal bilancio, la Parrocchia non è in grado di far fronte se non attraverso la stipulazione di un ulteriore finanziamento.

Per questo motivo il Consiglio Pastorale degli Affari Economici assieme a don Federico, invita i parrocchiani e tutta la comunità di Maserada ad un aiuto concreto per poter fra fronte a questa spesa, almeno per la parte delle opere strettamente necessarie ed urgenti.

\figuramedia{0.9\textwidth}{campane}
Per chi desiderasse, può contattare direttamente don Federico o la Parrocchia negli orari di apertura della Canonica di Maserada sul Piave.

Ringraziamo fin d'ora per la generosità dimostrata.

Il Consiglio Pastorale degli Affari Economici

