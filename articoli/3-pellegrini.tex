\section{Pellegrini di speranza condivisa}

Il giubileo è un evento religioso che si celebra ogni 25 anni, è un'occasione per la conversione, la riconciliazione e la celebrazione della misericordia di Dio.
È per questi validi motivi che noi pellegrini abbiamo deciso di partecipare col cuore pieno di speranza.

Il 17 Ottobre siamo partiti da Maserada con direzione Roma.
Ritrovo dal parcheggio della chiesa con i bagagli in spalla e l'adrenalina al pensiero del bellissimo viaggio che ci aspettava. Pronti, partenza, via, Roma ci aspetta!
Il viaggio seppur lungo è stato piacevole per la compagnia, allietata anche dal nostro Don Federico che ha creato un'atmosfera piacevole, unendoci tutti con un unico obiettivo: passare del tempo prezioso con il prossimo e con Dio.
Dopo una bella mangiata con i panini e bevute di vino siamo ripartiti verso la nostra prima destinazione: Basilica di Santa Maria Maggiore.

Abbiamo attraversato la prima Porta Santa del nostro pellegrinaggio e subito abbiamo visto la tomba di Papa Francesco.
La Basilica era meravigliosa e abbiamo ammirato la reliquia della culla del bambin Gesù.

Una foto di gruppo davanti alla facciata e poi siamo ripartiti verso Sacrofano, dove abbiamo cenato tutti assieme creando un momento un momento di condivisione con anche il gruppo di Varago e Candelù.

La stanchezza della lunga giornata si fa sentire, quindi tutti a coricarsi a letto, ma c'è chi ha ancora le energie per qualche partita a carte!

Il sabato è stato ricco di emozioni a partire dall'ingresso nella Basilica di San Pietro attraverso la Porta Santa, toccando con mano la grandezza di questo momento, l'attimo più importante di questo Giubileo. Dopo una visita all'interno abbiamo avuto la possibilità di osservare le Grotte Vaticane, mentre i giovani del gruppo hanno affrontato la salita a piedi al <<cupolone>> da cui si poteva godere di un cielo limpido con una visione unica di Roma dall'alto.

Nel pomeriggio c'è stato il tempo per ammirare i monumenti più importanti della città eterna e dopo una passeggiata su e giù per la Roma antica siamo risaliti in corriera con il nostro super autista che ci ha portati a Sacrofano, naturalmente anche quella sera non sono mancate le risate e giochi.

La domenica abbiamo deciso di prendere parte alla Santa Messa nella Basilica di San Paolo fuori le Mura, una delle 4 basiliche giubilari.

Una volta arrivati abbiamo trovato una sorpresa ad attenderci: la maratona di Roma, che passava proprio davanti all'ingresso, qui i più temerari hanno deciso di intrufolarsi nella massa di corridori, mentre gli altri, incitando i maratoneti hanno aspettato la fine del loro passaggio.

Siamo partiti per fare ritorno a casa, durante il viaggio tra canti, giochi e video-chiamate a Don Federico ci siamo divertiti davvero tanto ed è la gioia e la spensieratezza vissute in questi tre giorni che non ci mollavano più anche quando siamo scesi dalla corriera.

\figuramedia{0.6\textwidth}{note}

Questa non è la fine del nostro racconto, ma solo l'inizio del nostro pellegrinaggio. Roma non è la destinazione e Maserada non è il ritorno, bensì il nostro punto di partenza.

In tre giorni il sole ha scaldato i nostri cuori, il silenzio meditativo della preghiera ha curato la nostra anima, il vento ha spazzato i pregiudizi avvelenati che ci appesantivano e i sorrisi delle persone ci hanno cambiato gli occhi. È proprio guardando negli occhi dei nostri compagni che abbiamo imparato che la vita e tutto ciò che si possiede, non ha senso se non si hanno dei buoni amici per condividere.

La speranza che portiamo a casa è una nuova consapevolezza di avere il compito di fare il primo passo, anche se è difficile, anche se non siamo convinti, anche se non conosciamo la strada, anche se abbiamo timore. Dobbiamo ripartire e imparare di nuovo a camminare: facciamo tutti un passo e poi un altro, poi un altro ancora, fino a quando sapremo correre su una strada illuminata dal Suo amore, dalla pace che sapremo donare e dall'aiutare il prossimo come fossimo noi stessi e <<come l'avessimo fatto a Lui>>.

È questo insegnamento di Gesù, che più di tutto abbiamo appreso e messo in pratica, aiutati dal nostro sacerdote Don Federico e dai nostri compagni pellegrini. E da questo agire così semplice, quasi banale, abbiamo potuto vedere personalmente gli effetti travolgenti delle Sue parole: se apriamo il cuore agli altri e capiamo di essere tutti fratelli e sorelle, apriamo il cuore alla vita e in un gesto possiamo portare una rivoluzione di pace, in un passo cambiare la vita di un'altra persona.
E prima di tutto la nostra.
Siamo partiti come pellegrini di speranza, in cerca di una luce nuova da portare nelle nostre vite, e torniamo come pellegrini di pace, <<testimoni del suo amore>>.
La gioia tra tutte queste persone di età differenti è un'opera meravigliosa dello Spirito Santo che abbiamo colto a mani piene.

Auguriamo a tutta la comunità un sereno Natale del Signore e un felice nuovo anno.

\firma{I ragazzi del coro 9 Note\\ che hanno partecipato al Pellegrinaggio}


