\section{I cori uniti}

Due importanti eventi hanno caratterizzato l'autunno canoro del nostro coro: la partecipazione insieme alla corale San Giorgio alla Messa di ingresso del nostro nuovo Parroco Don Federico e la partecipazione alla Messa dei ``Cori d'accordo''.

Siamo stati chiamati ad animare la Messa d'ingresso di Don Federico come nuovo parroco di Maserada. Per noi è stato un grande piacere raccogliere l'invito del CPP. Certo è stato un bell'impegno riuscire ad unire le nostre voci con quelle della Scola Cantorum di San Giorgio, ma con la direzione d'insieme della maestra Michela e del maestro Alessandro abbiamo aiutato l'assemblea ad accompagnare l'ingresso di Don Federico nella nostra parrocchia.

La comunità si è sentita più unita ed è stata una bella festa. Come dice il Salmo 133: ``Ecco, com'è bello e com'è dolce che i fratelli vivano insieme''. 

\figuramedia{0.85\textwidth}{cori}

Gli otto cori delle parrocchie del vicariato di Spresiano si sono ritrovati insieme sabato 16 novembre per la Messa della chiesa di Santa Maria Assunta di Varago. Il nostro ``Coro 9 note'' insieme ad altri 7 ``Cori d'Accordo'' ha unito voci e strumenti non per un concerto, ma per celebrare l'Eucarestia come un grande popolo che canta.
È stata una Messa molto sentita e partecipata. Il nostro intento, come in quell'occasione, è di realizzare il sogno di come dovrebbe essere, tutte le domeniche, ogni assemblea: partecipe, viva, entusiasta e festosa! Auspichiamo con il nostro servizio, di essere testimoni della bellezza e della gioia del messaggio evangelico che siamo tutti chiamati ad annunciare.

Come ha scritto Sant'Agostino:
``Cantare è proprio di chi ama. Chi ha cantato di tutto cuore e con gioia, ama quel che ha cantato, ama il luogo in cui ha cantato, ama Colui per il quale ha cantato, ama, infine, coloro con i quali ha cantato.''

Quest'anno abbiamo dato anche il benvenuto a due nuovi cantori e una nuova strumentista; che il loro entusiasmo contagi altri giovani e bambini incuriositi e interessati a questo gioioso e coinvolgente servizio: non temete di venire a proporvi all'inizio o alla fine della Messa delle 10.30, noi vi accogliamo a braccia aperte!

Auguriamo a tutta la comunità un sereno Santo Natale del Signore e un nuovo anno ricco di pace e serenità.


\firma{il Coro 9 Note}

\vfill

\figuramedia{0.8\textwidth}{coretto}

