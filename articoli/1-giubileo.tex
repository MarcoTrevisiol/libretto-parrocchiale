\section{L'alfabeto del Giubileo}

\begin{description}
  \item[A di Anno Santo] Si è aperto il Giubileo, un tempo di grazia che ci è donato come opportunità di ricentrare la nostra vita nel Signore Gesù e fare esperienza della Sua infinita Misericordia, che avvolge e attraversa le nostre tante miserie. Lasciamoci provocare ancora una volta da questo Amore vivificante che sempre ci precede ed attende! Con Speranza, buon cammino!

  \item[B di Bolla] Ogni Giubileo viene proclamato tramite la pubblicazione di una Bolla Papale d'indizione. Per il 2025 Papa Francesco ha scritto Spes non confundit: la speranza non delude, parole riprese dall'apostolo Paolo che infonde coraggio alla comunità cristiana di Roma. Il Papa ci invita a ravvivare la speranza in Cristo: la nostra e quella delle persone che incontriamo nelle nostre vite!

  \item[C di Credere] <<Credere non è un'idea ma un incontro con Cristo che cambia il cuore>> così ha detto Papa Francesco in una sua udienza. Fede e speranza procedono insieme! In questo Anno Santo, Signore, donaci la grazia di esserti fedeli nella Speranza!

  \item[D di Diocesi] Anche nella nostra Diocesi è possibile conseguire ottenere l'Indulgenza Giubilare recandosi in pellegrinaggio in uno dei dieci luoghi sacri designati, attraverso il sacramento della Confessione, della Comunione e della preghiera del Sommo Pontefice.

  \item[E di Evangelizzare] Evangelizzare, per la Chiesa, è portare la Buona Novella in tutti gli strati dell'umanità. È, con il suo influsso, trasformare dal di dentro, rendere nuova l'umanità stessa: <<Ecco, io faccio nuove tutte le cose>>. Ma non c'è nuova umanità se prima non ci sono uomini nuovi, della novità del battesimo e della vita secondo il Vangelo.

  \item[F di Futuro] In Spes non confundit Papa Francesco parla di quanti guardano all'avvenire senza fiducia, ma noi viviamo nella <<certezza che niente e nessuno potrà mai separarci dall'amore divino>>. Allora com'è possibile pensare al futuro senza questa fiducia in Dio che non ci abbandona mai, che cammina fra noi? Questa certezza ci dona occhi nuovi con cui riscoprire le nostre vite e entusiasmo da trasmettere a quanti sono sfiduciati!

  \item[G di Grazia] La grazia è una partecipazione alla vita di Dio; ci introduce nell'intimità della vita trinitaria. Mediante il Battesimo il cristiano partecipa alla grazia di Cristo, Capo del suo corpo. Come <<figlio adottivo>>, egli può ora chiamare Dio <<Padre>>, in unione con il Figlio unigenito. Riceve la vita dello Spirito che infonde in lui la carità e forma la Chiesa.
\end{description}

\figurawrapdx{0.6\textwidth}{giubileo.jpg}
\paragraph{}
\vspace*{-\parskip}

\begin{description}

  \item[H di Hèset] Hèsed è un vocabolo ebraico, molto ricorrente nell'Antico Testamento, difficilmente traducibile con una sola parola italiana, in quanto si può intendere come tutti quei sentimenti che caratterizzano una relazione d'amore: tenerezza, bontà, misericordia, fedeltà. È la parola che definisce l'alleanza tra Dio e il suo popolo: un amore gratuito, incondizionato, permanente. In questo Anno Santo lasciamoci coinvolgere in questa relazione d'amore!
\end{description}

\begin{description}
  \item[I di Inno] Il Teologo e musicologo Sequeri, autore del testo dell'Inno del Giubileo Pellegrini di Speranza, così si è espresso: <<Un inno è sempre una sintesi della fede che anima un Giubileo, e ogni volta che viene suonato o cantato dai pellegrini, ne rinnova l'emozione e ne conserva la memoria>>. È un canto accompagnato dall'augurio che giunga alle orecchie di Colui che lo fa sgorgare. È Dio che, come fiamma sempre viva, tiene accesa la speranza e dà energia al passo del popolo che cammina.

  \item[L di Libertà] È un richiamo antico, nell'Anno Giubilare, quello di proclamare <<la liberazione nella terra per tutti i suoi abitanti>> (Lv 25,10). Ma di che libertà stiamo parlando? Non solo quella dalla schiavitù o i percorsi di reinserimento nella comunità per i detenuti, ma quella che tocca tutti noi: libertà di essere noi stessi, libertà di amare ed essere amati, libertà dalle nostre piccole o grandi dipendenze --- libertà di vivere con gioia!

  \item[M di Misericordia] <<Se tutto il nostro cristianesimo non ci porta alla misericordia, abbiamo sbagliato strada, perché la misericordia è l'unica vera meta di ogni cammino spirituale>>. -- La misericordia non è una dimensione fra le altre, ma è il centro della vita cristiana: non c'è cristianesimo senza misericordia -- è <<l'aria da respirare>>.
\end{description}
