\section{Formazione CPP CCP}

Sabato 13 gennaio, a Carbonera, si è tenuto il secondo incontro dei membri dei Consigli delle Collaborazioni Pastorali (CCP) e di quelli dei Consigli Pastorali Parrocchiali (CPP) del nostro Vicariato, nell’ambito dell’iniziativa di formazione promossa dalla Diocesi di Treviso e nata da alcuni momenti di ascolto e confronto nel Consiglio Pastorale Diocesano, nel Consiglio Presbiterale, con i Coordinatori delle Collaborazioni Pastorali e in altre occasioni.

Il primo incontro si era tenuto sabato 18 novembre, sempre a Carbonera.

L’invito del nostro vescovo Michele era di riprendere la scelta chiave del Cammino Sinodale diocesano (2018) al fine di valorizzare l’esperienza ecclesiale dei vari Consigli, affinché diventino sempre più luoghi di sinodalità e corresponsabilità, scuole di ascolto e di discernimento, promotori e animatori di comunità che sappiano passare “dall’autopreservazione all’uscita”.

È stata la seconda giornata di formazione per i membri dei Consigli pastorali parrocchiali e di Collaborazione, introdotta, come la precedente, dal nostro vescovo Michele Tomasi.

La scelta chiave del Cammino sinodale diocesano, vissuto tra il 2017 e il 2018, aveva messo al centro proprio gli organismi di partecipazione, una scommessa della Chiesa di Treviso, ha sottolineato il Vescovo, “sulla sua fede nei doni ricevuti da ciascuno e ciascuna, doni ricevuti nel Battesimo, che permettono la collaborazione nel corpo di Cristo che è la Chiesa, e la corresponsabilità nell’annuncio del Vangelo”.

Don Antonio Mensi, vicario per le Collaborazioni pastorali, e don Virgilio Sottana, direttore della Scuola diocesana di formazione teologica, hanno presentato e introdotto il percorso.

Tema dell’incontro è stato: “Com’è strutturata la Chiesa”; a svilupparlo si sono avvicendati come relatori don Daniele Fregonese e don Fabio Franchetto.

Don Daniele ha illustrato il ruolo della Chiesa partendo dall’evangelizzazione che fa nascere la comunità cristiana nel territorio, la sua articolazione in Diocesi e Parrocchia, il richiamo al n. 780 del Catechismo della Chiesa Cattolica “La Chiesa è in questo mondo il sacramento della salvezza, il segno e lo strumento della comunione di Dio e degli uomini” e l’organizzazione territoriale come strumento perché tutto sia dato a tutti.

Don Fabio, invece, ha affrontato nel dettaglio la struttura della Chiesa, dal concetto di sinodalità, alla sua costituzione gerarchica, al governo della Chiesa particolare, alla parrocchia e al ruolo del parroco, ai consigli parrocchiali con uno sguardo allo Statuto del C.P.P.

Dopo il pranzo condiviso, i presenti si sono suddivisi in alcuni laboratori per un confronto su diverse tematiche proposte: Compiti dei Consigli (CPP e CCP) in relazione tra loro. Relazione tra Consigli comunità cristiana e territorio. Sinodalità e processi decisionali e significato del “consigliare”. Percorsi di rinnovo dei consigli. Costruire le Collaborazioni Pastorali come “comunità di comunità”.

Come il primo, anche questo secondo appuntamento è stato vissuto in un sincero spirito di discernimento e confronto, nell’ascolto delle relazioni e nella partecipazione ai laboratori.

Molto intenso e apprezzato anche il momento del pranzo condiviso tra tutti i presenti all’incontro.

\firma{Mauro}

\figuramedia{0.9\textwidth}{disegno-giardino.jpeg}
