\section{Giubileo dei Giovani 2025}

Oltre 1 milione di giovani, 146 Paesi di provenienza, 3500 volontari, 7 giorni..i numeri del Giubileo dei Giovani non finiscono qui, ma bastano a darci l'idea della grandiosità dell'evento svoltosi a Roma dal 28 luglio al 3 agosto 2025. È stata una grande festa della fede che ha abbracciato la città di Roma travolgendola con la gioia, l'entusiasmo, i sorrisi di tantissimi giovani provenienti da tutto il Mondo. Per 7 giorni la metropolitana, gli autobus, le vie e le piazze hanno ascoltato canti, visto bandiere sventolare, abbracci, incontri e balli tra ragazzi fino a prima sconosciuti, perché è proprio questo lo spirito del Giubileo, condividere la propria fede e cultura di provenienza, e lasciarsi travolgere da quella altrui, in uno scambio che rende fratelli nel nome di Gesù.

Il Giubileo ha avuto inizio con la messa in Piazza San Pietro martedì pomeriggio, e già dall'ingresso nella piazza abbiamo avuto la percezione di essere attorniati dal Mondo con la curiosità di indovinare le bandiere dei Paesi attorno a noi. Alla messa è seguito, a sorpresa, l'arrivo ed il giro in papamobile di Papa Leone XIV: <<Speriamo che tutti voi siate sempre segni di speranza nel mondo!>>, ci ha detto durante il suo saluto, chiedendo a tutti noi di pregare per la pace e di esserne costruttori nei luoghi di vita. Al Giubileo hanno infatti partecipato anche molti gruppi di giovani provenienti da Paesi in conflitto, dall'Ucraina, alla Terra Santa, dalla Siria all'Iran, all'Iraq, al Ruanda, al Sud Sudan.

Mercoledì e giovedì il Giubileo si è spostato nelle parrocchie, piazze, teatri e periferie di Roma con eventi di preghiera, animazione, incontro, concerti e feste, a cura dei movimenti religiosi e associazioni con lo scopo di portare i giovani a vivere e sperimentare la fede nella quotidianità, scoprendo il messaggio di speranza che viene dal Vangelo. Giovedì pomeriggio tutti noi Italiano abbiamo vissuto in Piazza San Pietro l'evento <<Tu sei Pietro>>, il rito di professione di fede guidato dal Cardinale Zuppi. Ci ha esortato a disarmare i nostri cuori affinché le comunità diventino case di pace, <<piccole ma mai mediocri, grandi perché umili, libere perché legate dall'amore, capaci di lavorare gli uni per gli altri e di pensarsi insieme. Perché, anche le più piccole sono sempre grandi se dentro c'è il Signore e possiamo fare grandi cose. E, infine, l'invito a confessare la fede sia individualmente che insieme, per sostenersi a vicenda, attingere alla fraternità, all'amicizia. Volersi bene perché l'amore ripara, ripara tutto, sempre, molto più di quello che crediamo>>. Il rito è stato preceduto da uno spettacolo musicale e di testimonianze sul significato della speranza.

Venerdì la giornata è stata dedicata al sacramento della Riconciliazione nello scenario del Circo Massimo, o nei vari oratori e basiliche con anche l'attraversamento della Porta Santa.

Sabato e domenica sono stati i giorni centrali e conclusivi del Giubileo in cui ci siamo tutti radunati nell'immensa spianata di Tor Vergata, come 25 anni fa in occasione della Giornata Mondiale della Gioventù con Giovanni Paolo II. Come sentinelle del mattino ci siamo messi in cammino con lo zaino in spalla per raggiungere il luogo della veglia e la celebrazione con il Papa, le strade si sono riempite di allegria e festa nonostante il caldo, la fatica e la stanchezza. Arrivati a Tor Vergata ci siamo sistemati in un settore, e tra balli, canti, giochi a carte, conversazioni in inglese e spagnolo abbiamo atteso l'arrivo del Papa che prima con l'elicottero ha sorvolato sopra di noi, e poi con la papamobile ha attraversato i settori. Emozionante è stato vedere come moltissimi ragazzi, carichi di entusiasmo, corressero tra i settori per rincorrere la papamobile per cercare il saluto del Papa o lasciargli la loro bandiera. L'euforia, la festa e la musica hanno lasciato spazio al silenzio e alla preghiera durante l'adorazione eucaristica, in un momento carico di emozione, fede e speranza che solo la presenza di Dio può rendere possibile. Ti senti parte di qualcosa di grande che è difficile da spiegare, percepisci l'abbraccio del Signore nel silenzio e nella condivisione con migliaia di giovani nella fede. Sono state poste al Papa tre domande, la prima riguardava il valore e significato dell'amicizia nel nostro tempo: <<oggi ci sono algoritmi che ci dicono quello che dobbiamo vedere, quello che dobbiamo pensare, e quali dovrebbero essere i nostri amici. E allora le nostre relazioni diventano confuse, a volte ansiose. È che quando lo strumento domina sull'uomo, l'uomo diventa uno strumento: sì, strumento di mercato, merce a sua volta. Solo relazioni sincere e legami stabili fanno crescere storie di vita buona.>>

La seconda domanda chiedeva dove trovare il coraggio per scegliere e compiere scelte radicali: <<Il coraggio per scegliere viene dall'amore, che Dio ci manifesta in Cristo. San Giovanni Paolo II disse: ``è Gesù che cercate quando sognate la felicità; è Lui che vi aspetta quando niente vi soddisfa di quello che trovate; è Lui la bellezza che tanto vi attrae; è Lui che vi provoca con quella sete di radicalità che non vi permette di adattarvi al compromesso; è Lui che vi spinge a deporre le maschere che rendono falsa la vita; è Lui che vi legge nel cuore le decisioni più vere che altri vorrebbero soffocare''. Ecco scelte radicali, scelte piene di significato: il matrimonio, l'ordine sacro, e la consacrazione religiosa esprimono il dono di sé, libero e liberante, che ci rende davvero felici. E lì troviamo la felicità: quando impariamo a donare noi stessi, a donare la vita per gli altri.>>

La terza chiedeva come incontrare Dio ed essere sicuri della sua presenza anche nelle difficoltà: <<l'amico che sempre accompagna la nostra coscienza è Gesù. Volete incontrare veramente il Signore Risorto? Ascoltate la sua parola, che è Vangelo di salvezza! Cercate la giustizia, rinnovando il modo di vivere, per costruire un mondo più umano! Servite il povero, testimoniando il bene che vorremmo sempre ricevere dal prossimo! Perciò incontriamo veramente Cristo nella Chiesa. Quanto ha bisogno il mondo di missionari del Vangelo che siano testimoni di giustizia e di pace! Quanto ha bisogno il futuro di uomini e donne che siano testimoni di speranza! Ecco, carissimi giovani, il compito che il Signore Risorto ci consegna.>>

La Veglia ha lasciato poi seguito alla notte trascorsa sotto i sacchi a pelo, con qualche goccia di pioggia, fino all'alba del nuovo giorno in cui abbiamo celebrato la Santa Messa con il Papa che ci ha esortato ad aspirare <<a cose grandi, alla santità, ovunque siate. Non accontentatevi di meno. Allora vedrete crescere ogni giorno, in voi e attorno a voi, la luce del Vangelo. Vi affido a Maria, la Vergine della speranza. Con il suo aiuto, tornando nei prossimi giorni ai vostri Paesi, in tutte le parti del mondo, continuate a camminare con gioia sulle orme del Salvatore, e contagiate chiunque incontrate col vostro entusiasmo e con la testimonianza della vostra fede! Buon cammino!>>

È stata poi l'occasione per dare l'appuntamento per la prossima Giornata Mondiale della Gioventù, a Seoul, in Sud Corea nel 2027.

Rimessi gli zaini in spalla, abbiamo ripreso il cammino per ritornare alle nostre case, consapevoli di aver vissuto un'esperienza indimenticabile, di aver respirato una chiesa giovane e con la speranza nel cuore che come ci ha detto Papa Leone <<la pienezza della nostra esistenza non dipende da ciò che accumuliamo né da ciò che possediamo, ma è legata piuttosto a ciò che con gioia sappiamo accogliere e condividere.>>

\firma{Lucia P.}

