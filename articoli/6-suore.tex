\section{una piccola luce}

Le suore Dorotee all'interno della parrocchia di Maserada sono solo una piccola luce, l'età e le poche forze residue non permettono attività apostolica diretta, come lo è stata per tantissimi anni.

\figurawrap{0.5\textwidth}{lanterna}

Questa situazione concreta non spegne però quel fuoco che il nostro Fondatore, il Beato Don Luca Passi, voleva ardesse nel cuore di ogni sua figlia. Era solito ripetere ``Chi non arde non accende''. Lo diceva alle suore ed anche alle persone laiche che prima delle suore avevano accolto il suo Carisma e che operavano nell'Opera di S. Dorotea, una Associazione laicale a servizio delle giovani generazioni nata a Calcinate BG) nel 1815, (le suore le fonda nel 1838 perché fossero l'anima di questa realtà, che si stava diffondendo in tutta la penisola).

L'Associazione dell'Opera di S. Dorotea è presente da diversi anni anche a Maserada attraverso un gruppo di donne, che sentono il desiderio di farsi compagne di viaggio di ogni persona attraverso piccoli gesti quotidiani di carità.


Ma che cosa era e cosa è quest'Opera? È un'opera di carità spirituale.

``Se può sembrare facile tratteggiare che cos'è un'opera di'misericordia corporale' di fronte ad un uomo che ha fame, che è ammalato, che è prigioniero, calpestato nei suoi diritti fondamentali, più complesso risulta descrivere un'opera puramente spirituale. Tale è l'Opera di S. Dorotea che rimanda ai bisogni di altro genere propri dell'uomo: essere perdonato, consolato, istruito, consigliato, corretto, riscattato nella propria dignità di persona (\dots)

Oggetto dell'Opera è ``comunicare lo Spirito di Dio'' risvegliare la consapevolezza di essere figlio (cfr. Gal 4, 5-7) e prendersi a cuore il compito di crescere e far crescere in tale identità ``nella misura che conviene alla piena maturità di Cristo'' (Ef 4,13) (da ``Passi con l'Opera'')


La carità spirituale possiamo dire che non esige grandi forze fisiche, ma tanto amore, attenzione, ascolto e preghiera e in questo ognuna di noi si sente ancora attiva.

L'età non spegne il Carisma, ma permette di trovare altre forme per viverlo e renderlo vivo. Vorremmo perciò che la nostra comunità fosse quel fuoco o quella piccola luce a cui ogni persona può avvicinarsi per trovare uno spazio di ascolto e un luogo di preghiera.

Con l'augurio che possiate accostarvi alla vera luce che è il Signore Gesù vi diciamo

Buon Natale!

\firma{Le suore Dorotee}
