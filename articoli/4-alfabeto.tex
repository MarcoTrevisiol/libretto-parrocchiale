\section{L'alfabeto del Giubileo}
\begin{center}
\textit{da un'idea dell'unità pastorale di Pieve di Soligo}
\end{center}

\begin{description}
	\item[U di Universale] La speranza del Giubileo non ha confini: è un dono che Dio offre a tutti, in ogni angolo del mondo. È un invito che parla ogni lingua e raggiunge ogni cuore, perché l'amore di Dio è davvero universale. Non importa il colore della pelle, la cultura, l'età o il passato di ciascuno: siamo tutti figli di un unico Padre.

	Il Giubileo ci ricorda che la Chiesa è casa aperta, dove nessuno è straniero e tutti possono sentirsi accolti. In un mondo che spesso divide, il Vangelo unisce.

	\item[V di Vangelo] Il Vangelo è la Buona Notizia. La vita umana, ci assicura il Vangelo, non può non essere che una ricerca di felicità.

	<<In che cosa consiste la buona, la bella, e beata notizia? È l'annuncio che è possibile vivere bene, vivere meglio, per tutti. È possibile avere la vita in pienezza. La felicità è possibile. Anzi, vicina. Gesù ne possiede la chiave. Gesù non ci ha dato una teoria religiosa, un sistema di pensiero. Ci ha comunicato vita ed ha creato in noi l'anelito verso una vita più grande>> (P. Giovanni Vannucci).

	Anche in questo periodo di vacanze, il Vangelo meditato e vissuto quotidianamente allarga il nostro orizzonte, rinvigorisce corpo e mente, mette in noi uno Spirito nuovo.  A noi, discepoli e missionari, la gioia e la bellezza di raccontare Gesù come si racconta una storia d'amore!"


	\item[Z di Zakhar] Zākhār in ebraico significa <<ricordare>> nel senso di rendere presente ciò che è stato, conservandolo vivo nell'alleanza. Nella storia Dio ricorda spesso l'alleanza stretta con il suo popolo, è un ricordo con delle conseguenze attive anche nel nostro presente.

	Papa Francesco, in Evangelii Gaudium, scrive: <<La gioia del Vangelo si fonda su una memoria grata>> (EG 13). L'Eucaristia che celebriamo è zākhār: è memoria viva, efficace, che trasforma l'oggi di ognuno di noi."
\end{description}
