\section{Ingresso don Federico}

\figuramedia{0.5\textwidth}{don-federico-1}

Grande partecipazione di volontari per l'ingresso di Don Federico. L'affetto dei Maseradesi si è fatto sentire con un'organizzazione impeccabile corredata dalla splendida collaborazione anche dei fratelli di Varago e Candelù.

\figuramedia{0.5\textwidth}{don-federico-2}

Domenica 20 l'appuntamento per i volontari è stato alle 9.00. C'era da allestire la chiesa e il sagrato con nastri e fiocchi, la canonica per l'arrivo del Vescovo, l'Oratorio con tavole e festoni e vivande per il rinfresco, l'Auditorium da spolverare e la Chiesetta da ordinare.

\figuramedia{0.5\textwidth}{don-federico-3}

Alle 15.30 si sono dati appuntamento in piazzetta San Francesco, Don Federico e i parrocchiani di Varago e Candelù, orgogliosi di accompagnare in processione verso la chiesa il loro sacerdote, come un dono all'intera comunità e per stargli vicino di fronte a un così grande impegno.

\figuramedia{0.5\textwidth}{don-federico-4}

Verificare la privacy della foto dei bambini con i palloncini, se è da oscurare i volti per pubblicarla.
In alternativa c'è quella sotto più ampia.

\figuramedia{0.5\textwidth}{don-federico-5}

Alla chiesetta della Madonna delle Vittorie erano ad attendere la processione i paesani di Maserada, e una fila di bambini della scuola materna con in mano un palloncino bianco. Alla vista dei bambini Don Federico ha esordito con: ``cari bambini so che avete una sorpresa per me'' ebbene, uno dei giovanissimi ha risposto: ``la sorpresa siamo noi!''. Don Federico è esploso in una piacevole risata. La chiesetta è stata una tappa importante dove don Federico ha chiesto alla Madonna la protezione per il nuovo incarico. Ecco che dopo un breve discorso, una volta data la benedizione agli astanti, la processione è proseguita verso la canonica.

Qui, ad attendere Don Federico, c'era il Vescovo Michele che ha seguito le pratiche burocratiche e il giuramento del nuovo parroco.

Dalla Canonica, Don Federico accompagnato dal Vescovo e da alcuni preti amici e conoscenti, ha raggiunto il sagrato della Chiesa, ulteriore tappa per incontrare le autorità di Maserada ovvero il sindaco e gli assessori comunali. Qui un discorso ufficiale del sindaco ha accolto il nuovo sacerdote ringraziandolo per l'importante incarico.

In chiesa la cerimonia ha seguito il ricco protocollo di segni e messaggi di rito. I parrocchiani hanno gremito la chiesa in ogni spazio, e inoltre è stato possibile seguire la funzione anche in Auditorium, dove i giovani hanno organizzato una diretta streaming con anche la possibilità di fare la comunione.

Dopo la cerimonia in chiesa, ecco che in Oratorio c'era ad attendere tutti i parrocchiani delle tre frazioni un meraviglioso rinfresco, organizzato sotto un grande capannone che ha coperto il campo da basket pronto per assolvere la sua funzione anche in caso di maltempo. Atteso l'arrivo del Vescovo si è proceduti al taglio del nastro ufficiale del rinfresco, e una folla oceanica e affamata si è servita in abbondanza con panini, tramezzini, pizzette, dolci bibite e vino.

Un sottofondo musicale ha accompagnato gli astanti con simpatici interventi, mentre uno stuolo di volontari era indaffarato a distribuire le vivande e intrattenere le persone. Il vescovo Michele si è rivelato veramente simpaticissimo sempre pronto alla battuta, alla riflessione, alle fotografie di gruppo ai selfie e persino a un piacevolissimo intermezzo musicale al termine della serata, dove abbracciata una chitarra ha suonato alcune famose canzoni popolari accompagnato dai presenti.

\firma{Si ringraziano i tanti volontari e in particolare del Gruppo Oratorio, Catechiste, Mamme GrEst, Alpini, Gruppo Pesca e Gruppo San Francesco, il CPP e Manuel Mazzaro per la regia dell'evento.}

