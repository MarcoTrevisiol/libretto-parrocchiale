\section{Gruppo Caritas interparrocchiale Candelù – Maserada  - Varago \\ Santo Natale 2025}

Domenica 16 Novembre si è celebrata la nona Giornata Mondiale dei Poveri, istituita da Papa Francesco nel 2016

La giornata che ha coinciso con il “Giubileo dei Poveri”, e il Papa Leone XIV ha esortato tutti, credenti e non, a riflettere sulla speranza che si manifesta attraverso la carità e la solidarietà  per offrire segni concreti di aiuto a chi è nel bisogno.

Nelle nostre comunità non dobbiamo considerare Poveri  solo coloro che economicamente non riescono ad avere un tenore di vita dignitoso. Ci sono molti altri tipi di Povertà che ci circondano e che per indifferenza, fretta o peggio ancora e volutamente cerchiamo di non vedere per non esserne coinvolti.  Ci sono Persone chesoffrono di solitudine, Persone senza lavoro, Persone in carcere, Persone ammalate, Persone o famiglie sfrattate che non trovano più  alloggio, Persone extracomunitarie.…… Per tutte queste categorie di Povertà  ed altre ancora ogni credente deve prendere coscienza della loro presenza  e lasciarsi coinvolgere dalle loro situazioni  offrendo un po’ del proprio tempo ascoltando, consigliando, realizzando iniziative che aiutino concretamente  a superare le loro difficoltà, ma sempre con disponibilità, col sorriso ed un cuore aperto ad accettarle senza nessun pre-giudizio nei loro confronti.

La nostra cultura attuale, purtroppo, ha messo ai primi posti  l’egoismo personale, il potere e la ricchezza sacrificando la dignità delle persone sull’altare dei beni materiali, togliendo così lo sguardo dalle povertà che ci circondano.

Il Signore ha detto “Quando avrete  aiutato un Povero è come l’aveste fatto a me e sarete ricompensati in eterno”

Alcuni  dati sulla  Caritas Interparrocchiale di Candelù-Maserada-Varago

Contatti : Candelù–Nadia 3483505136 Maserada–Maurizio 3476606942 Varago–Rosita  3462284633

Centro di Ascolto : Le persone o famiglie che aiutiamo sono accolte dalle volontarie del CdA inizialmente per il primo colloquio di conoscenza con l’iscrizione dei dati significativi del nucleo famigliare e la consegna della tessera ritiro alimenti che viene rinnovata ogni 4 mesi verificando nuovamente  la loro situazione famigliare. Il tutto in accordo con le Assistenti  Sociali del  Comune.  Contatti: Paola  3334744592

Vestiario : Ritiriamo vestiario previo  contatto telefonico.  Preghiamo che i capi di vestiario che ci vengono consegnati, se pur usati, siano in buono stato e puliti. Coloro che ricevono questi capi sono persone come noi che dobbiamo dignitosamente rispettare.  Contatti: Manuela 3930790246

Borse Alimenti: Consegniamo alimenti di prima necessità  come pasta, latte, zucchero, riso, farina 00, biscotti, passata, scatolame vario, prodotti per l’igiene personale e per la casa, secondo le esigenze e le scorte disponibili. Alimenti che in parte  riusciamo ad acquistare direttamente con le offerte di privati e con il contributo Comunale, altri vengono raccolti con la Colletta Solidale, altri ancora sono raccolti nelle chiese o in occasioni delle varie attività di catechismo (1\textasciicircum Comunione o Cresima). Altri alimenti venivano ritirati alla Caritas Tarvisina ma queste consegne si sono interrotte definitivamente alla fine del 30/09/2025

SONO 31 I NUCLEI FAMIGLIARI AIUTATI DALLA CARITAS INTERPARROCCHIALE NEL 2025

Di cui : Adulti nr. 50 (32 italiani-18 stranieri) – Minori nr. 23 (7 italiani-16 stranieri) =Totale nr. 73 Persone

Borse Spesa : Consegna ogni 15 giorni – Numero Borse consegnate a fine anno 2025 Totale nr. 663                                                         1° Tipo – Nr. 22 Nuclei Famigliari composti  da 1 o 2 persone –    Tot. Borse nr. 458                                                                          2° Tipo – Nr.  8  Nuclei Famigliari composti  da 3 o più persone – Tot. Borse nr. 205 

L’apertura della sede Caritas è ogni 15 giorni al Mercoledì : Mattino 9,00-11,00 Pomeriggio 15,30-17,30

\firma{a tutte  le  comunità auguriamo\\
Un felice  santo natale ed un\\
Sereno anno nuovo 2026}

