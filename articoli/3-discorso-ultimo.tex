\subsection{\dots e a Don Mirco a nome di tutta la parrocchia}

Caro don Mirco,

grazie per la Sua presenza oggi in mezzo a noi, grande è la nostra gioia nel rivederla per celebrare e ringraziare il Signore di un percorso lungo 23 anni. Un periodo di grazia in mezzo a noi. 

Dio l’ha scelto e mandato per predicare l’amore nel più ampio senso del termine. Stiamo attraversando un periodo storico difficile per la società e anche per la chiesa. Sicuramente in tante delicate situazioni non deve essere stato facile per Lei, ma ha sempre permesso allo Spirito Santo di permearla affinché si facesse strumento per confortare, guidare e indirizzare tanti di noi che ne hanno avuto bisogno.

Un grazie ai doni che il Padre Le ha riversato quali l’umiltà, la bontà, la discrezione, la magnanimità, la capacità di ascolto e di conciliazione nelle diverse visioni, e di perdono che hanno facilitato il Suo agire nel dialogo con le varie realtà parrocchiali e civili, con le quali si è sempre confrontato per i problemi comuni. La devozione alla Madonna delle Vittorie è stata per Lei un faro e guida nel Suo cammino, sicuramente l’ha sostenuta nell’affrontare il quotidiano, a gioire con noi nelle occasioni di festa e a consolarci nei momenti del dolore.

Un grazie speciale per la Sua presenza nella comunità delle nostre suore dorotee, riconoscendo la loro indispensabile presenza, e sostenendo la loro fede con la messa quotidiana. 

Grande il lavoro materiale per gli innumerevoli progetti e opere fatte, che hanno richiesto sforzo, ma sostenute dalla Sua determinazione sono state realizzate, tra le quali la chiesa, la canonica, l’oratorio, l’auditorium e i locali della Caritas, ambito questo dove il suo sostegno e la Sua presenza sono stati fondamentali.

Grazie, per tutti coloro, soprattutto i più fragili, gli anziani, i malati, i disabili, i sofferenti che nelle sue parole e visite a casa o in ospedale, hanno trovato sollievo e consolazione. 

Importante e significativa la Sua presenza nelle nostre famiglie, con le quali amava trascorrere momenti di convivialità e di allegria, e come guida spirituale dei gruppi famiglie.

Grazie per la presenza con i bambini e giovani, per la disponibilità nel catechismo e nei campi scuola, dove la ricordiamo con la simpatia e l’allegria nei bans, e nelle serate vissute con le note della sua chitarra o del pianoforte, quale grande appassionato di musica. E’ stato un instancabile guida nelle escursioni in montagna. Ricordiamo con gioia anche i pellegrinaggi da Lei organizzati e le visite ai presepi, una tappa caratteristica della nostra comunità.

Ha predicato l’amore come il Padre Le aveva chiesto quale testimone credibile, guidandoci alla fede e ai valori cristiani con grande impegno, dedizione e generosità e per questo amato da tutti noi. Le tappe cristiane della nostra vita e di quella dei nostri figli sono piene di Suoi ricordi e del Suo sorriso. Grande è stata per noi la Sua testimonianza di fede e affidamento al Padre, come abbiamo di recente festeggiato nel suo 50° di ordinazione sacerdotale.

Noi le assicuriamo il nostro ricordo e le nostre preghiere, affinchè il Signore continui a guidarla e a renderla testimone nel nuovo nuovo percorso di vita che si apre ora davanti a Lei.

Siamo certi che anche Lei si ricorderà di noi, e che ci sosterrà nelle Sue preghiere per il bene e il cammino della nostra comunità.

Come segno di ringraziamento Le facciamo dono di questo quadro realizzato da Enzo Barbon, come simbolo di pastore in mezzo a noi, e nel ricordo indelebile che ha lasciato alla nostra comunità. 


Grazie di cuore don Mirco, rimarrà sempre nei nostri cuori.

