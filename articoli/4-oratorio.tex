\section{Oratorio Rinnovato}

\figuramedia{\textwidth}{auditorium-1.jpg}

\figurawrap{0.5\textwidth}{auditorium-2.jpg}

Si chiama ``Oratorio Rinnovato'' la nuova iniziativa nata a fine aprile con l’obbiettivo di effettuare una serie di opere ed interventi di sistemazione ed ammodernamento nel nostro Oratorio. Volontari appartenenti al Gruppo Pesca, al Gruppo Oratorio, Catechiste, GrEst, e altri ancora, hanno deciso di incontrarsi, unire le loro forze, e portare avanti una sequenza di interventi approvati da Don Mirco. La prima opera concordata è stato il rifacimento del fondo e del colore della facciata dell’Oratorio sopra la scalinata dell’Auditorium e anche a lato dalla parte dell’ingresso piccolo e del campo da basket.
Inoltre anche l’androne d’ingresso dell’Auditorium è stato tinteggiato internamente. Per fare tutto questo sono state montate apposite impalcature.

Queste opere in realtà non erano state previste all’inizio ma sono state aggiunte in corsa per sfruttare l’occasione del rifacimento totale della scalinata dell’Auditorium progettata dal CPAE. Anche le luci dell’illuminazione esterna sono state sostituite con nuove lampade moderne. La scalinata andava rifatta perché fortemente danneggiata ed ora, a lavori ultimati, possiamo dire che l’ingresso di Oratorio e Auditorium sono tornati ad essere molto belli!  La prossima attività ``di Rinnovo'' pianificata è il rifacimento della tinta di tutte le stanze dell’Oratorio e contestualmente la manutenzione degli impianti elettrici. Si inizierà dopo il 15 luglio e si andrà avanti completando una sola stanza alla volta per garantire lo svolgersi delle attività pastorali che ripartiranno in autunno.

\figuramedia{0.95\textwidth}{auditorium-3.jpg}


In Oratorio dal 24 giugno al 12 luglio è in corso il GrEst che vede iscritti 140 ragazzini dalla prima elementare alla seconda media. Ci sono ben 60 animatori e 37 mamme che dalle 15.00 alle 18.00 li accompagnano nei consueti momenti di preghiera, laboratori e tanti giochi e balli. Al mercoledì c’è la consueta giornata in piscina ad Oderzo ma anche la nuova uscita al Sinapsi Park di Nervesa. Quest’anno per la prima volta per i 20 ragazzi di terza media viene sperimentato un nuovo laboratorio creativo appositamente ideato per loro che sono anche aiuto animatori.



Gli animatori prima dell’inizio del GrEst hanno partecipato ad alcuni momenti formativi assieme ai coetanei di Varago, Breda e Candelù.

\figurawrap{0.55\textwidth}{auditorium-4.jpg}


I volontari che seguono l’Oratorio sono sempre super impegnati per la manutenzione ordinaria e straordinaria, la cura del verde, la pulizia e l’ordine.



Dobbiamo assolutamente ringraziare tutte le persone che dedicano il loro tempo e anche le loro risorse per il bene dell’Oratorio dell'Auditorium. Molti Gruppi e associazioni inoltre si adoperano direttamente per l’acquisto di materiale di consumo e giochi per i tanti ragazzi che frequentano l’Oratorio e fra questi segnaliamo questa volta il Gruppo Pesca e la Pro Loco. Ringraziamo anche le tante famiglie che ospitiamo per festeggiare i compleanni dei loro bambini e che contribuiscono anche loro al funzionamento della struttura.
Insomma è sempre piacevole testimoniare in questo consueto articolo di luglio la passione e l’amore di tante persone che appartengono alla “famiglia dell’Oratorio” e che vogliono bene alla parrocchia e a Maserada. Grazie di cuore a tutti e aspettiamo anche nuovi amici e amiche che abbiano piacere di contribuire ad un... ``Oratorio Rinnovato''!
 
\firma{Buona estate da Don Mirco e Gruppo Oratorio}
