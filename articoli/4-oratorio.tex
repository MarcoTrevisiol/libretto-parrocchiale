\section{Oratorio San Giorgio}
\vspace{-1ex}
{\Large\scshape{Centro della nostra Parrocchia}}

Nel nostro Oratorio abbiamo il catechismo, le riunioni pastorali, le prove dei cori, gli scout, i compleanni, le riunioni condominiali, il GrEst, il corso di pittura, i tornei sportivi, la festa solenne di San Giorgio con i giochi all'aperto, la Castagnata di ottobre e tanto, tanto altro. A supporto di tutte queste attività ci sono i volontari che regalano il loro tempo per il bene della parrocchia di Maserada.

Esiste il Gruppo delle Catechiste, il Gruppo delle Mamme GrEst, il Gruppo Pulizia che ordina a lava tutte le stanze, il Gruppo San Giorgio che si occupa della manutenzione ordinaria e straordinaria della struttura: pota gli alberi, taglia l'erba, sistema i giochi all'esterno, esegue riparazioni, coordina le feste di compleanno e supporta alla bisogna tutte le attività pastorali e gli eventi.

Di recente si è costituito anche il Gruppo Oratorio Rinnovato formato da alcuni artigiani che hanno appena terminato un bellissimo compito ovvero tinteggiare tutte le stanze dell'oratorio e modernizzare il loro impianto di illuminazione.

Tanti altri lavori sono pendenti come riverniciare la casetta di legno all'esterno, ripassare le linee del campo da basket, ripristinare l'impianto di irrigazione attorno al campo da calcio, sistemare il terreno e il manto erboso davanti alle porte dove si sono formati due grandi avvallamenti, cambiare la rete alta dietro le porte perché rovinata, cambiare e aggiungere delle nuove giostre per i bimbi, eseguire le opere di manutenzione in canonica e tanti ma tanti altri oneri. L'oratorio e i suoi volontari collaborano anche con i gruppi e le associazioni del territorio, le suore Dorotee, l'Asilo di Maserada, la scuola, l'Auditorium e le istituzioni comunali.

Sono allo studio molti progetti per la collettività, i giovani, le famiglie, il divertimento, ma per poterli realizzare servirebbe l'aiuto di nuovi volontari in particolare i giovani genitori dei tanti ragazzini che frequentano la struttura. Chi ha a cuore il bene dei propri figli o nipoti e avesse piacere di essere protagonista nel realizzare nuove opere, iniziative e idee e fare gruppo e <<costruire comunità>> si faccia avanti e ci contatti. Con il sorriso e l'impegno potremo fare tante belle cose\ldots

Vi aspettiamo, contattateci!

\firma{Buona Pasqua dal gruppo Oratorio San Giorgio di Maserada}
