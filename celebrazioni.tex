\noindent{\begin{center}
\fbox{
\begin{minipage}{0.95\textwidth}
    \begin{center}
    \textbf{\textit{CELEBRAZIONI per la Madonna delle Vittorie - 2024}}
    \end{center}
\end{minipage}
}
\end{center}}

% \vspace*{\fill}
\small

\begin{center}
\begin{tblr}
{
    rows = {valign = m},
    column{1} = {8em, c},
    column{2} = {0.75\textwidth, l},
    hlines,
    hline{1,Z} = {1pt},
    colsep=2pt,
    rowsep=3pt
}

{Domenica\\ 7 luglio} &
{
Ore 7.45. Partenza dal santuario per portare l’icona
della Madonna in chiesa, dove alle ore 8.00 ci sarà
la S. Messa.\\
Ore 10.30. Seconda S. Messa.\\
Ore 16.30. Vesperi e breve Adorazione eucaristica.\\
Ore 19.00 S. Messa vespertina.
}
\\
Lunedì, martedì, mercoledì
&
{
Triduo in preparazione della festa.\\
Ore 20.30. S. Messa è alla sera.
}
\\
{Giovedì\\ 11 luglio}
&
{
Ore 9.30 S. Messa per tutti, nella quale verrà dato il
sacramento dell’Unzione degli infermi per anziani e
persone particolarmente sofferenti.
}
\\
{Venerdì\\ 12 luglio}
&
{
Ore 17.30, Adorazione eucaristica.\\
Ore 18.30, S. Messa.
}
\\
{Sabato\\ 13 luglio}
&
{
Confessioni ore 9.00-11.30 e 16.00-18.30 con un
confessore straordinario.
}
\\
{Domenica\\ 14 luglio}
&
{
Ore 8.00 e 10.30. SS. Messe.\\
La Messa seconda, presieduta da un vescovo e
animata dal nostro Coro “S. Giorgio”, si concluderà
con la processione della Madonna che verrà
riportata nel santuario.\\
Ore 16.00. Qui verrà recitato il santo rosario.
}
\\
{Lunedì\\ 15 luglio}
&
{
Ore 8.00 e 9.00. Messa in santuario.
}
\\
{Martedì\\ 16 luglio}
&
{
Ore 8.30. Messa in santuario.
}
\end{tblr}

\end{center}


\normalsize
