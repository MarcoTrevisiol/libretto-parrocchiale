\noindent{\begin{center}
\fbox{
\begin{minipage}{0.95\textwidth}
    \begin{center}
    \textbf{\textit{CELEBRAZIONI per la Santa PASQUA - 2024}}
    \end{center}
\end{minipage}
}
\end{center}}

\vspace*{\fill}
\small

\begin{center}
\begin{tblr}
{
    rows = {valign = m},
    column{1} = {8em, c},
    column{2} = {0.75\textwidth, l},
    hlines,
    hline{1,Z} = {1pt},
    colsep=2pt,
    rowsep=3pt
}

Domenica 24 marzo
&
{
{\large\textbf{\textit{Domenica delle Palme}}}\\
Ore 8.00. S. Messa.\\
Ore 10.15. Benedizione degli ulivi nella piazzetta dell'Emigrante. Segue S. Messa della Passione.\\
Ore 16.00. Vesperi e apertura dell'adorazione eucaristica delle \textit{Quarantore}.
}
\\
Lunedì 25, martedì 26, mercoledì 27 marzo
&
{
Ore 9.00-11.45, 15.00-18.25. Adorazione di Gesù eucaristico nell'ostensorio.\\
Ore 18.30. S. Messa.
}
\\
&
{\large\textbf{\textit{Triduo Pasquale}}} \\
Giovedì 28 marzo
&
{
{\large\textit{Giovedì Santo}} \\
Giorno dell'ultima cena di Gesù con gli apostoli. \\
Ore 9.30. S. Messa del Sacro Crisma in Duomo a Treviso. Segue a pag. \pageref{sacro-crisma}. \label{celebrazioni} \\
Ore 7.30. Preghiera delle \textit{lodi} in chiesa. \\
Ore 15.00-19.00. Confessioni. \\
Ore 19.30. S. Messa in \textit{Coena Domini} per tutti. \\
Segue un'ora di adorazioni.
}
\\
Venerdì 29 marzo
&
{
{\large\textit{Venerdì Santo}} \\
Giorno della Passione di Gesù sulla Croce. \\
È il giorno di digiuno e astinenza.\\
Ore 7.30. Preghiera con i salmi di mattutino e lodi. \\
Ore 15.00. Via crucis per bambini, ragazzi, mamme, nonni e altri. \\
Ore 18.30. Azione liturgica della Passione di Gesù.\\
Ore 20.30. La processione sarà fatta con la collaborazione delle 7 parrocchie solo a San Bartolomeo.
}
\\
{Sabato 30 \\ marzo}
&
{
{\large\textit{Sabato Santo}} \\
Giorno del lutto e del silenzio. \\
Ore 7.30. Preghiera con i salmi mattutino e lodi. \\
Ore 9.00-11.50, 14.30-18.30. Confessioni. \\
Ore 21.00. Grande veglia pasquale, S. Messa della notte di Pasqua.
}
\\
Domenica 31 marzo
&
{
{\large\textbf{\textit{Santa Pasqua}}} \\
Giorno della resurrezione di Nostro Signore Gesù Cristo. \\
Ore 8.00, 9.30, 11.00. SS. Messe. \\
Ore 16.30. Vesperi solenni e adorazione eucaristica.
}
\\
{Lunedì 1 \\ aprile}
&
{
{\large\textit{Lunedì dell'Angelo}} \\
Ore 9.00. S. Messa.
}
\end{tblr}

\end{center}


\normalsize
