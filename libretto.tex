\documentclass[a5paper,10pt]{article} % Starts an article
\usepackage[utf8]{inputenc}
\usepackage[left=1.5cm,right=1.5cm,top=2cm,bottom=3cm]{geometry}
\usepackage[center]{titlesec}
\usepackage{fouriernc}
\usepackage{ragged2e}
\usepackage{parskip}
\titleformat{\section}
 {\Huge\scshape\filcenter}{\thesection}{0pt}{\titlerule}[\titlerule]

\setcounter{secnumdepth}{0}
\title{Costruire Comunità} % Title
\author{}

\begin{document} % Begins a document
  \maketitle
  
  \clearpage 
\begin{justify}
\section{Gli auguri di Don Mirco}
Cari Maseradesi,

siamo giunti ormai a Natale, con la sua atmosfera di festa di serenità e di speranza, nonostante ci siano ancora guerre nel mondo.

Il Natale di Cristo, ci riconduce al centro del mistero che segna uno spartiacque nel cammino verso l'incontro con Dio. Per molti anni gli uomini hanno cercato il suo volto, si sono posti in ascolto di una sua parola. Ed ecco Dio ci è venuto incontro, il cielo si è aperto ed  accaduto un straordinario evento, nella maniera pi impensabile per l'uomo. In una grotta poverissima,  nato il bambino divino, avvolto in fasce e deposto in una mangiatoia. Un segno normale attesta che Dio  venuto, ha assunto la fragilit della nostra carne, come segno del suo essere fra di noi, come rivelatore del Padre. I nostri schemi sono svaniti, perch Dio  presente nellinfinitamente piccolo, la sua parola un vagito di neonato, che si affida a un volto che gli sorrida, a una mano che lo accarezzi, a un seno che lo nutra. Contempliamo estasiati questo mistero che suscita tenerezza, amore gioia e festa per tutti. Cari amici, Gesù, desidera essere il centro di queste feste che suscitano sentimenti di bontà, di fraternit, di giustizia e di pace.

Accogliendo Lui nella nostra vita quotidiana, riceveremo la sua grazia che rinnova il nostro spirito e lo rende capace di rispettare laltro, di amarlo, di costruire ponti e abbattere le barriere dellegoismo, dellodio e della violenza.

Vi auguro, con la semplicit dei pastori, di andare incontro a Gesù, che ama essere aiutato nei poveri, nei fragili, negli ammalati, negli anziani. Lui  presente nel suo pane eucaristico per donarci ristoro e grazia e letizia da offrire agli sfiduciati. Prepariamoci a questo evento di gioia accostandoci, come afferma papa Francesco, anche al sacramento del perdono, che riempie il nostro cuore della sua luce e del suo amore e cancella i nostri peccati.

Auguro a tutti voi un santo Natale incontrando Gesù e un nuovo anno di serenit e salute, con la forza divina del suo Spirito.
\end{justify}
\begin{flushright}
    Il vostro parroco Don Mirco
\end{flushright}

\end{document}

