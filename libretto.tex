\documentclass[a5paper,10pt]{article} % Starts an article
\usepackage[utf8]{inputenc}
\usepackage[left=1.5cm,right=1.5cm,top=2cm,bottom=3cm]{geometry}
\usepackage[center]{titlesec}
\usepackage{librebaskerville}
\usepackage{array}
\usepackage{graphicx}
\graphicspath{ {./images/} }

\titleformat{\section}
 {\huge\scshape\filcenter}{\thesection}{0pt}{\titlerule}[\titlerule]

\setcounter{secnumdepth}{0}
\title{} % Title
\author{}

\begin{document} % Begins a document
  \maketitle
  
  \clearpage
  \noindent{\begin{center}
\fbox{
\begin{minipage}{0.95\textwidth}
    \begin{center}
    \textbf{\textit{CELEBRAZIONI per la Santa PASQUA - 2024}}
    \end{center}
\end{minipage}
}
\end{center}}

\small

\begin{center}
\begin{tblr}
{
    rows = {valign = m},
    column{1} = {8em, c},
    column{2} = {0.75\textwidth, l},
    hlines,
    hline{1,Z} = {1pt},
    colsep=2pt,
    rowsep=1.5pt
}

Domenica 24 marzo
&
{
\textbf{\textit{Domenica delle Palme}}\\
Ore 8.00. S. Messa.\\
Ore 10.15. Benedizione degli ulivi nella piazzetta dell'Emigrante. Segue S. Messa della Passione.\\
Ore 16.00. Vesperi e apertura dell'adorazione eucaristica delle \emph{Quarantore}.
}
\\
Lunedì 25, martedì 26, mercoledì 27 marzo
&
{
Ore 9.00-11.45, 15.00-18.25. Adorazione di Gesù eucaristico nell'ostensorio.\\
Ore 18.30. S. Messa.
}
\\
&
\textbf{\textit{Triduo Pasquale}} \\
Giovedì 28 marzo
&
{
\textit{Giovedì Santo} \\
Giorno dell'ultima cena di Gesù con gli apostoli. \\
I sacerdoti al mattino sono in Duomo con il vescovo per la consacrazione del Sacro Crisma, dell'olio dei catecumeni e degli infermi, e rinnovano le loro promesse sacerdotali. \\
Ore 7.30. Preghiera delle \emph{lodi} in chiesa. \\
Ore 15.00-19.00. Confessioni. \\
Ore 19.30. S. Messa in \emph{Coena Domini} per tutti. \\
Segue un'ora di adorazioni.
}
\\
Venerdì 29 marzo
&
{
\textit{Venerdì Santo} \\
Giorno della Passione di Gesù sulla Croce. \\
È il giorno importante di digiuno e astinenza.\\
Ore 7.30. Preghiera con i salmi di mattutino e lodi. \\
Ore 15.00. Via crucis per bambini, ragazzi, mamme, nonni e altri. \\
Ore 19.00. Azione liturgica della Passione di Gesù. La processione sarà fatta con la collaborazione delle 7 parrocchie solo in XYZ.
}
\\
{Sabato 30 \\ marzo}
&
{
\textit{Sabato Santo} \\
Giorno del lutto e del silenzio. \\
Ore 7.30. Preghiera con i salmi mattutino e lodi. \\
Ore 9.00-11.50, 14.30-18.30. Confessioni. \\
Ore 21.00. Grande veglia pasquale, S. Messa della notte di Pasqua.
}
\\
Domenica 31 marzo
&
{
\textbf{\textit{Santa Pasqua}} \\
Giorno della resurrezione di Nostro Signore Gesù Cristo. \\
Ore 8.00, 9.30, 11.00. SS. Messe. \\
Ore 16.30. Vesperi solenni e adorazione eucaristica.
}
\\
{Lunedì 1 \\ aprile}
&
{
\textit{Lunedì dell'Angelo} \\
Ore 9.00. S. Messa.
}
\end{tblr}

\end{center}


\normalsize

  \clearpage
  
\section{Gli auguri di Don Mirco}
Cari Maseradesi,

siamo giunti ormai a Natale, con la sua atmosfera di festa di serenità e di speranza, nonostante ci siano ancora guerre nel mondo.

Il Natale di Cristo, ci riconduce al centro del mistero che segna uno spartiacque nel cammino verso l'incontro con Dio. Per molti anni gli uomini hanno cercato il suo volto, si sono posti in ascolto di una sua parola. Ed ecco Dio ci è venuto incontro, il cielo si è aperto ed  accaduto un straordinario evento, nella maniera pi impensabile per l'uomo. In una grotta poverissima,  nato il bambino divino, avvolto in fasce e deposto in una mangiatoia. Un segno normale attesta che Dio  venuto, ha assunto la fragilit della nostra carne, come segno del suo essere fra di noi, come rivelatore del Padre. I nostri schemi sono svaniti, perch Dio  presente nellinfinitamente piccolo, la sua parola un vagito di neonato, che si affida a un volto che gli sorrida, a una mano che lo accarezzi, a un seno che lo nutra. Contempliamo estasiati questo mistero che suscita tenerezza, amore gioia e festa per tutti. Cari amici, Ges, desidera essere il centro di queste feste che suscitano sentimenti di bontà, di fraternit, di giustizia e di pace.

Accogliendo Lui nella nostra vita quotidiana, riceveremo la sua grazia che rinnova il nostro spirito e lo rende capace di rispettare laltro, di amarlo, di costruire ponti e abbattere le barriere dellegoismo, dellodio e della violenza.

Vi auguro, con la semplicit dei pastori, di andare incontro a Gesù, che ama essere aiutato nei poveri, nei fragili, negli ammalati, negli anziani. Lui  presente nel suo pane eucaristico per donarci ristoro e grazia e letizia da offrire agli sfiduciati. Prepariamoci a questo evento di gioia accostandoci, come afferma papa Francesco, anche al sacramento del perdono, che riempie il nostro cuore della sua luce e del suo amore e cancella i nostri peccati.

Auguro a tutti voi un santo Natale incontrando Gesù e un nuovo anno di serenit e salute, con la forza divina del suo Spirito.

Il vostro parroco don Mirco

\end{document}

